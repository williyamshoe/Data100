
% Default to the notebook output style

    


% Inherit from the specified cell style.




    
\documentclass[11pt]{article}

    
    
    \usepackage[T1]{fontenc}
    % Nicer default font (+ math font) than Computer Modern for most use cases
    \usepackage{mathpazo}

    % Basic figure setup, for now with no caption control since it's done
    % automatically by Pandoc (which extracts ![](path) syntax from Markdown).
    \usepackage{graphicx}
    % We will generate all images so they have a width \maxwidth. This means
    % that they will get their normal width if they fit onto the page, but
    % are scaled down if they would overflow the margins.
    \makeatletter
    \def\maxwidth{\ifdim\Gin@nat@width>\linewidth\linewidth
    \else\Gin@nat@width\fi}
    \makeatother
    \let\Oldincludegraphics\includegraphics
    % Set max figure width to be 80% of text width, for now hardcoded.
    \renewcommand{\includegraphics}[1]{\Oldincludegraphics[width=.8\maxwidth]{#1}}
    % Ensure that by default, figures have no caption (until we provide a
    % proper Figure object with a Caption API and a way to capture that
    % in the conversion process - todo).
    \usepackage{caption}
    \DeclareCaptionLabelFormat{nolabel}{}
    \captionsetup{labelformat=nolabel}

    \usepackage{adjustbox} % Used to constrain images to a maximum size 
    \usepackage{xcolor} % Allow colors to be defined
    \usepackage{enumerate} % Needed for markdown enumerations to work
    \usepackage{geometry} % Used to adjust the document margins
    \usepackage{amsmath} % Equations
    \usepackage{amssymb} % Equations
    \usepackage{textcomp} % defines textquotesingle
    % Hack from http://tex.stackexchange.com/a/47451/13684:
    \AtBeginDocument{%
        \def\PYZsq{\textquotesingle}% Upright quotes in Pygmentized code
    }
    \usepackage{upquote} % Upright quotes for verbatim code
    \usepackage{eurosym} % defines \euro
    \usepackage[mathletters]{ucs} % Extended unicode (utf-8) support
    \usepackage[utf8x]{inputenc} % Allow utf-8 characters in the tex document
    \usepackage{fancyvrb} % verbatim replacement that allows latex
    \usepackage{grffile} % extends the file name processing of package graphics 
                         % to support a larger range 
    % The hyperref package gives us a pdf with properly built
    % internal navigation ('pdf bookmarks' for the table of contents,
    % internal cross-reference links, web links for URLs, etc.)
    \usepackage{hyperref}
    \usepackage{longtable} % longtable support required by pandoc >1.10
    \usepackage{booktabs}  % table support for pandoc > 1.12.2
    \usepackage[inline]{enumitem} % IRkernel/repr support (it uses the enumerate* environment)
    \usepackage[normalem]{ulem} % ulem is needed to support strikethroughs (\sout)
                                % normalem makes italics be italics, not underlines
    

    
    
    % Colors for the hyperref package
    \definecolor{urlcolor}{rgb}{0,.145,.698}
    \definecolor{linkcolor}{rgb}{.71,0.21,0.01}
    \definecolor{citecolor}{rgb}{.12,.54,.11}

    % ANSI colors
    \definecolor{ansi-black}{HTML}{3E424D}
    \definecolor{ansi-black-intense}{HTML}{282C36}
    \definecolor{ansi-red}{HTML}{E75C58}
    \definecolor{ansi-red-intense}{HTML}{B22B31}
    \definecolor{ansi-green}{HTML}{00A250}
    \definecolor{ansi-green-intense}{HTML}{007427}
    \definecolor{ansi-yellow}{HTML}{DDB62B}
    \definecolor{ansi-yellow-intense}{HTML}{B27D12}
    \definecolor{ansi-blue}{HTML}{208FFB}
    \definecolor{ansi-blue-intense}{HTML}{0065CA}
    \definecolor{ansi-magenta}{HTML}{D160C4}
    \definecolor{ansi-magenta-intense}{HTML}{A03196}
    \definecolor{ansi-cyan}{HTML}{60C6C8}
    \definecolor{ansi-cyan-intense}{HTML}{258F8F}
    \definecolor{ansi-white}{HTML}{C5C1B4}
    \definecolor{ansi-white-intense}{HTML}{A1A6B2}

    % commands and environments needed by pandoc snippets
    % extracted from the output of `pandoc -s`
    \providecommand{\tightlist}{%
      \setlength{\itemsep}{0pt}\setlength{\parskip}{0pt}}
    \DefineVerbatimEnvironment{Highlighting}{Verbatim}{commandchars=\\\{\}}
    % Add ',fontsize=\small' for more characters per line
    \newenvironment{Shaded}{}{}
    \newcommand{\KeywordTok}[1]{\textcolor[rgb]{0.00,0.44,0.13}{\textbf{{#1}}}}
    \newcommand{\DataTypeTok}[1]{\textcolor[rgb]{0.56,0.13,0.00}{{#1}}}
    \newcommand{\DecValTok}[1]{\textcolor[rgb]{0.25,0.63,0.44}{{#1}}}
    \newcommand{\BaseNTok}[1]{\textcolor[rgb]{0.25,0.63,0.44}{{#1}}}
    \newcommand{\FloatTok}[1]{\textcolor[rgb]{0.25,0.63,0.44}{{#1}}}
    \newcommand{\CharTok}[1]{\textcolor[rgb]{0.25,0.44,0.63}{{#1}}}
    \newcommand{\StringTok}[1]{\textcolor[rgb]{0.25,0.44,0.63}{{#1}}}
    \newcommand{\CommentTok}[1]{\textcolor[rgb]{0.38,0.63,0.69}{\textit{{#1}}}}
    \newcommand{\OtherTok}[1]{\textcolor[rgb]{0.00,0.44,0.13}{{#1}}}
    \newcommand{\AlertTok}[1]{\textcolor[rgb]{1.00,0.00,0.00}{\textbf{{#1}}}}
    \newcommand{\FunctionTok}[1]{\textcolor[rgb]{0.02,0.16,0.49}{{#1}}}
    \newcommand{\RegionMarkerTok}[1]{{#1}}
    \newcommand{\ErrorTok}[1]{\textcolor[rgb]{1.00,0.00,0.00}{\textbf{{#1}}}}
    \newcommand{\NormalTok}[1]{{#1}}
    
    % Additional commands for more recent versions of Pandoc
    \newcommand{\ConstantTok}[1]{\textcolor[rgb]{0.53,0.00,0.00}{{#1}}}
    \newcommand{\SpecialCharTok}[1]{\textcolor[rgb]{0.25,0.44,0.63}{{#1}}}
    \newcommand{\VerbatimStringTok}[1]{\textcolor[rgb]{0.25,0.44,0.63}{{#1}}}
    \newcommand{\SpecialStringTok}[1]{\textcolor[rgb]{0.73,0.40,0.53}{{#1}}}
    \newcommand{\ImportTok}[1]{{#1}}
    \newcommand{\DocumentationTok}[1]{\textcolor[rgb]{0.73,0.13,0.13}{\textit{{#1}}}}
    \newcommand{\AnnotationTok}[1]{\textcolor[rgb]{0.38,0.63,0.69}{\textbf{\textit{{#1}}}}}
    \newcommand{\CommentVarTok}[1]{\textcolor[rgb]{0.38,0.63,0.69}{\textbf{\textit{{#1}}}}}
    \newcommand{\VariableTok}[1]{\textcolor[rgb]{0.10,0.09,0.49}{{#1}}}
    \newcommand{\ControlFlowTok}[1]{\textcolor[rgb]{0.00,0.44,0.13}{\textbf{{#1}}}}
    \newcommand{\OperatorTok}[1]{\textcolor[rgb]{0.40,0.40,0.40}{{#1}}}
    \newcommand{\BuiltInTok}[1]{{#1}}
    \newcommand{\ExtensionTok}[1]{{#1}}
    \newcommand{\PreprocessorTok}[1]{\textcolor[rgb]{0.74,0.48,0.00}{{#1}}}
    \newcommand{\AttributeTok}[1]{\textcolor[rgb]{0.49,0.56,0.16}{{#1}}}
    \newcommand{\InformationTok}[1]{\textcolor[rgb]{0.38,0.63,0.69}{\textbf{\textit{{#1}}}}}
    \newcommand{\WarningTok}[1]{\textcolor[rgb]{0.38,0.63,0.69}{\textbf{\textit{{#1}}}}}
    
    
    % Define a nice break command that doesn't care if a line doesn't already
    % exist.
    \def\br{\hspace*{\fill} \\* }
    % Math Jax compatability definitions
    \def\gt{>}
    \def\lt{<}
    % Document parameters
    \title{hw5}
    
    
    

    % Pygments definitions
    
\makeatletter
\def\PY@reset{\let\PY@it=\relax \let\PY@bf=\relax%
    \let\PY@ul=\relax \let\PY@tc=\relax%
    \let\PY@bc=\relax \let\PY@ff=\relax}
\def\PY@tok#1{\csname PY@tok@#1\endcsname}
\def\PY@toks#1+{\ifx\relax#1\empty\else%
    \PY@tok{#1}\expandafter\PY@toks\fi}
\def\PY@do#1{\PY@bc{\PY@tc{\PY@ul{%
    \PY@it{\PY@bf{\PY@ff{#1}}}}}}}
\def\PY#1#2{\PY@reset\PY@toks#1+\relax+\PY@do{#2}}

\expandafter\def\csname PY@tok@w\endcsname{\def\PY@tc##1{\textcolor[rgb]{0.73,0.73,0.73}{##1}}}
\expandafter\def\csname PY@tok@c\endcsname{\let\PY@it=\textit\def\PY@tc##1{\textcolor[rgb]{0.25,0.50,0.50}{##1}}}
\expandafter\def\csname PY@tok@cp\endcsname{\def\PY@tc##1{\textcolor[rgb]{0.74,0.48,0.00}{##1}}}
\expandafter\def\csname PY@tok@k\endcsname{\let\PY@bf=\textbf\def\PY@tc##1{\textcolor[rgb]{0.00,0.50,0.00}{##1}}}
\expandafter\def\csname PY@tok@kp\endcsname{\def\PY@tc##1{\textcolor[rgb]{0.00,0.50,0.00}{##1}}}
\expandafter\def\csname PY@tok@kt\endcsname{\def\PY@tc##1{\textcolor[rgb]{0.69,0.00,0.25}{##1}}}
\expandafter\def\csname PY@tok@o\endcsname{\def\PY@tc##1{\textcolor[rgb]{0.40,0.40,0.40}{##1}}}
\expandafter\def\csname PY@tok@ow\endcsname{\let\PY@bf=\textbf\def\PY@tc##1{\textcolor[rgb]{0.67,0.13,1.00}{##1}}}
\expandafter\def\csname PY@tok@nb\endcsname{\def\PY@tc##1{\textcolor[rgb]{0.00,0.50,0.00}{##1}}}
\expandafter\def\csname PY@tok@nf\endcsname{\def\PY@tc##1{\textcolor[rgb]{0.00,0.00,1.00}{##1}}}
\expandafter\def\csname PY@tok@nc\endcsname{\let\PY@bf=\textbf\def\PY@tc##1{\textcolor[rgb]{0.00,0.00,1.00}{##1}}}
\expandafter\def\csname PY@tok@nn\endcsname{\let\PY@bf=\textbf\def\PY@tc##1{\textcolor[rgb]{0.00,0.00,1.00}{##1}}}
\expandafter\def\csname PY@tok@ne\endcsname{\let\PY@bf=\textbf\def\PY@tc##1{\textcolor[rgb]{0.82,0.25,0.23}{##1}}}
\expandafter\def\csname PY@tok@nv\endcsname{\def\PY@tc##1{\textcolor[rgb]{0.10,0.09,0.49}{##1}}}
\expandafter\def\csname PY@tok@no\endcsname{\def\PY@tc##1{\textcolor[rgb]{0.53,0.00,0.00}{##1}}}
\expandafter\def\csname PY@tok@nl\endcsname{\def\PY@tc##1{\textcolor[rgb]{0.63,0.63,0.00}{##1}}}
\expandafter\def\csname PY@tok@ni\endcsname{\let\PY@bf=\textbf\def\PY@tc##1{\textcolor[rgb]{0.60,0.60,0.60}{##1}}}
\expandafter\def\csname PY@tok@na\endcsname{\def\PY@tc##1{\textcolor[rgb]{0.49,0.56,0.16}{##1}}}
\expandafter\def\csname PY@tok@nt\endcsname{\let\PY@bf=\textbf\def\PY@tc##1{\textcolor[rgb]{0.00,0.50,0.00}{##1}}}
\expandafter\def\csname PY@tok@nd\endcsname{\def\PY@tc##1{\textcolor[rgb]{0.67,0.13,1.00}{##1}}}
\expandafter\def\csname PY@tok@s\endcsname{\def\PY@tc##1{\textcolor[rgb]{0.73,0.13,0.13}{##1}}}
\expandafter\def\csname PY@tok@sd\endcsname{\let\PY@it=\textit\def\PY@tc##1{\textcolor[rgb]{0.73,0.13,0.13}{##1}}}
\expandafter\def\csname PY@tok@si\endcsname{\let\PY@bf=\textbf\def\PY@tc##1{\textcolor[rgb]{0.73,0.40,0.53}{##1}}}
\expandafter\def\csname PY@tok@se\endcsname{\let\PY@bf=\textbf\def\PY@tc##1{\textcolor[rgb]{0.73,0.40,0.13}{##1}}}
\expandafter\def\csname PY@tok@sr\endcsname{\def\PY@tc##1{\textcolor[rgb]{0.73,0.40,0.53}{##1}}}
\expandafter\def\csname PY@tok@ss\endcsname{\def\PY@tc##1{\textcolor[rgb]{0.10,0.09,0.49}{##1}}}
\expandafter\def\csname PY@tok@sx\endcsname{\def\PY@tc##1{\textcolor[rgb]{0.00,0.50,0.00}{##1}}}
\expandafter\def\csname PY@tok@m\endcsname{\def\PY@tc##1{\textcolor[rgb]{0.40,0.40,0.40}{##1}}}
\expandafter\def\csname PY@tok@gh\endcsname{\let\PY@bf=\textbf\def\PY@tc##1{\textcolor[rgb]{0.00,0.00,0.50}{##1}}}
\expandafter\def\csname PY@tok@gu\endcsname{\let\PY@bf=\textbf\def\PY@tc##1{\textcolor[rgb]{0.50,0.00,0.50}{##1}}}
\expandafter\def\csname PY@tok@gd\endcsname{\def\PY@tc##1{\textcolor[rgb]{0.63,0.00,0.00}{##1}}}
\expandafter\def\csname PY@tok@gi\endcsname{\def\PY@tc##1{\textcolor[rgb]{0.00,0.63,0.00}{##1}}}
\expandafter\def\csname PY@tok@gr\endcsname{\def\PY@tc##1{\textcolor[rgb]{1.00,0.00,0.00}{##1}}}
\expandafter\def\csname PY@tok@ge\endcsname{\let\PY@it=\textit}
\expandafter\def\csname PY@tok@gs\endcsname{\let\PY@bf=\textbf}
\expandafter\def\csname PY@tok@gp\endcsname{\let\PY@bf=\textbf\def\PY@tc##1{\textcolor[rgb]{0.00,0.00,0.50}{##1}}}
\expandafter\def\csname PY@tok@go\endcsname{\def\PY@tc##1{\textcolor[rgb]{0.53,0.53,0.53}{##1}}}
\expandafter\def\csname PY@tok@gt\endcsname{\def\PY@tc##1{\textcolor[rgb]{0.00,0.27,0.87}{##1}}}
\expandafter\def\csname PY@tok@err\endcsname{\def\PY@bc##1{\setlength{\fboxsep}{0pt}\fcolorbox[rgb]{1.00,0.00,0.00}{1,1,1}{\strut ##1}}}
\expandafter\def\csname PY@tok@kc\endcsname{\let\PY@bf=\textbf\def\PY@tc##1{\textcolor[rgb]{0.00,0.50,0.00}{##1}}}
\expandafter\def\csname PY@tok@kd\endcsname{\let\PY@bf=\textbf\def\PY@tc##1{\textcolor[rgb]{0.00,0.50,0.00}{##1}}}
\expandafter\def\csname PY@tok@kn\endcsname{\let\PY@bf=\textbf\def\PY@tc##1{\textcolor[rgb]{0.00,0.50,0.00}{##1}}}
\expandafter\def\csname PY@tok@kr\endcsname{\let\PY@bf=\textbf\def\PY@tc##1{\textcolor[rgb]{0.00,0.50,0.00}{##1}}}
\expandafter\def\csname PY@tok@bp\endcsname{\def\PY@tc##1{\textcolor[rgb]{0.00,0.50,0.00}{##1}}}
\expandafter\def\csname PY@tok@fm\endcsname{\def\PY@tc##1{\textcolor[rgb]{0.00,0.00,1.00}{##1}}}
\expandafter\def\csname PY@tok@vc\endcsname{\def\PY@tc##1{\textcolor[rgb]{0.10,0.09,0.49}{##1}}}
\expandafter\def\csname PY@tok@vg\endcsname{\def\PY@tc##1{\textcolor[rgb]{0.10,0.09,0.49}{##1}}}
\expandafter\def\csname PY@tok@vi\endcsname{\def\PY@tc##1{\textcolor[rgb]{0.10,0.09,0.49}{##1}}}
\expandafter\def\csname PY@tok@vm\endcsname{\def\PY@tc##1{\textcolor[rgb]{0.10,0.09,0.49}{##1}}}
\expandafter\def\csname PY@tok@sa\endcsname{\def\PY@tc##1{\textcolor[rgb]{0.73,0.13,0.13}{##1}}}
\expandafter\def\csname PY@tok@sb\endcsname{\def\PY@tc##1{\textcolor[rgb]{0.73,0.13,0.13}{##1}}}
\expandafter\def\csname PY@tok@sc\endcsname{\def\PY@tc##1{\textcolor[rgb]{0.73,0.13,0.13}{##1}}}
\expandafter\def\csname PY@tok@dl\endcsname{\def\PY@tc##1{\textcolor[rgb]{0.73,0.13,0.13}{##1}}}
\expandafter\def\csname PY@tok@s2\endcsname{\def\PY@tc##1{\textcolor[rgb]{0.73,0.13,0.13}{##1}}}
\expandafter\def\csname PY@tok@sh\endcsname{\def\PY@tc##1{\textcolor[rgb]{0.73,0.13,0.13}{##1}}}
\expandafter\def\csname PY@tok@s1\endcsname{\def\PY@tc##1{\textcolor[rgb]{0.73,0.13,0.13}{##1}}}
\expandafter\def\csname PY@tok@mb\endcsname{\def\PY@tc##1{\textcolor[rgb]{0.40,0.40,0.40}{##1}}}
\expandafter\def\csname PY@tok@mf\endcsname{\def\PY@tc##1{\textcolor[rgb]{0.40,0.40,0.40}{##1}}}
\expandafter\def\csname PY@tok@mh\endcsname{\def\PY@tc##1{\textcolor[rgb]{0.40,0.40,0.40}{##1}}}
\expandafter\def\csname PY@tok@mi\endcsname{\def\PY@tc##1{\textcolor[rgb]{0.40,0.40,0.40}{##1}}}
\expandafter\def\csname PY@tok@il\endcsname{\def\PY@tc##1{\textcolor[rgb]{0.40,0.40,0.40}{##1}}}
\expandafter\def\csname PY@tok@mo\endcsname{\def\PY@tc##1{\textcolor[rgb]{0.40,0.40,0.40}{##1}}}
\expandafter\def\csname PY@tok@ch\endcsname{\let\PY@it=\textit\def\PY@tc##1{\textcolor[rgb]{0.25,0.50,0.50}{##1}}}
\expandafter\def\csname PY@tok@cm\endcsname{\let\PY@it=\textit\def\PY@tc##1{\textcolor[rgb]{0.25,0.50,0.50}{##1}}}
\expandafter\def\csname PY@tok@cpf\endcsname{\let\PY@it=\textit\def\PY@tc##1{\textcolor[rgb]{0.25,0.50,0.50}{##1}}}
\expandafter\def\csname PY@tok@c1\endcsname{\let\PY@it=\textit\def\PY@tc##1{\textcolor[rgb]{0.25,0.50,0.50}{##1}}}
\expandafter\def\csname PY@tok@cs\endcsname{\let\PY@it=\textit\def\PY@tc##1{\textcolor[rgb]{0.25,0.50,0.50}{##1}}}

\def\PYZbs{\char`\\}
\def\PYZus{\char`\_}
\def\PYZob{\char`\{}
\def\PYZcb{\char`\}}
\def\PYZca{\char`\^}
\def\PYZam{\char`\&}
\def\PYZlt{\char`\<}
\def\PYZgt{\char`\>}
\def\PYZsh{\char`\#}
\def\PYZpc{\char`\%}
\def\PYZdl{\char`\$}
\def\PYZhy{\char`\-}
\def\PYZsq{\char`\'}
\def\PYZdq{\char`\"}
\def\PYZti{\char`\~}
% for compatibility with earlier versions
\def\PYZat{@}
\def\PYZlb{[}
\def\PYZrb{]}
\makeatother


    % Exact colors from NB
    \definecolor{incolor}{rgb}{0.0, 0.0, 0.5}
    \definecolor{outcolor}{rgb}{0.545, 0.0, 0.0}



    
    % Prevent overflowing lines due to hard-to-break entities
    \sloppy 
    % Setup hyperref package
    \hypersetup{
      breaklinks=true,  % so long urls are correctly broken across lines
      colorlinks=true,
      urlcolor=urlcolor,
      linkcolor=linkcolor,
      citecolor=citecolor,
      }
    % Slightly bigger margins than the latex defaults
    
    \geometry{verbose,tmargin=1in,bmargin=1in,lmargin=1in,rmargin=1in}
    
    

    \begin{document}
    
    
    \maketitle
    
    

    
    Before you turn in the homework, make sure everything runs as expected.
To do so, select \textbf{Kernel}\(\rightarrow\)\textbf{Restart \& Run
All} in the toolbar above. Remember to submit both on \textbf{DataHub}
and \textbf{Gradescope}.

Please fill in your name and include a list of your collaborators below.

    \begin{Verbatim}[commandchars=\\\{\}]
{\color{incolor}In [{\color{incolor}1}]:} \PY{n}{NAME} \PY{o}{=} \PY{l+s+s2}{\PYZdq{}}\PY{l+s+s2}{William Sheu}\PY{l+s+s2}{\PYZdq{}}
        \PY{n}{COLLABORATORS} \PY{o}{=} \PY{l+s+s2}{\PYZdq{}}\PY{l+s+s2}{\PYZdq{}}
\end{Verbatim}


    \begin{center}\rule{0.5\linewidth}{\linethickness}\end{center}

    Table of Contents{}

{Hypothesis Testing: Does The Hot Hand Effect Exist?}

{The Data}

{Problem 1 {[}5pts{]}}

{Problem 2 {[}5pts{]}}

{The Hypothesis}

{Understanding the Data}

{Problem 3 {[}1pt{]}}

{Problem 4 {[}2pts{]}}

{Problem 5 {[}2pts{]}}

{Problem 6 {[}10pts{]}}

{Problem 7 {[}5pts{]}}

{Defining a Test Statistic}

{Problem 8 {[}10pts{]}}

{Problem 9 {[}10pts{]}}

{A Different Statistic}

{Problem 10 {[}10pts{]}}

{Worked examples}

{Example 1}

{Example 2}

{Check your understanding}

{Computing the Expectated Value of \(T_{1, make}\)}

{Thought Exercise}

{Problem 11 {[}5pts{]}}

{Problem 12 {[}5pts{]}}

{The "Tversky Statistic" for Hot Hand Detection}

{Problem 13 {[}5pts{]}}

{Problem 14 {[}5pts{]}}

{Statistically Testing the Null Hypothesis}

{Problem 15 {[}10pts{]}}

{Problem 16 {[}Graded in the Synthesis Portion{]}}

{Synthesis}

{Data Generation Model {[}8pts{]}}

{Null Hypothesis {[}5pts{]}}

{Test Statistic {[}2pts{]}}

{Results {[}20pts{]}}

{Further Reading}

    \begin{Verbatim}[commandchars=\\\{\}]
{\color{incolor}In [{\color{incolor}2}]:} \PY{k+kn}{from} \PY{n+nn}{pathlib} \PY{k}{import} \PY{n}{Path}
        \PY{k+kn}{import} \PY{n+nn}{json}
        \PY{k+kn}{import} \PY{n+nn}{pandas} \PY{k}{as} \PY{n+nn}{pd}
        \PY{k+kn}{import} \PY{n+nn}{re}
        \PY{k+kn}{import} \PY{n+nn}{numpy} \PY{k}{as} \PY{n+nn}{np}
        \PY{k+kn}{import} \PY{n+nn}{itertools}
        \PY{k+kn}{import} \PY{n+nn}{matplotlib}\PY{n+nn}{.}\PY{n+nn}{pyplot} \PY{k}{as} \PY{n+nn}{plt}
        \PY{k+kn}{import} \PY{n+nn}{seaborn} \PY{k}{as} \PY{n+nn}{sns}
\end{Verbatim}


    \section{Hypothesis Testing: Does The Hot Hand Effect
Exist?}\label{hypothesis-testing-does-the-hot-hand-effect-exist}

\textbf{Due Date: Tuesday, November 20, 2018 at 11:59pm}

This homework concerns the game of basketball. If you're unfamiliar with
basketball, the first minute of
\href{https://www.youtube.com/watch?v=wYjp2zoqQrs}{this youtube video}
does a pretty good job of giving you the basic idea.

In basketball, the "hot hands effect" is a supposed phenomenon in which
a person who makes several successful baskets in a row experiences a
greater probability of scoring in further attempts. For example, a
player who has "made" three sucessful baskets in a row is considered to
have a higher probability of making a 4th basket than if they had just
missed a shot. In this assignment, we'll use 0 to represent a missed
basket and 1 to represent a made basket. Restating the hot hands effect
in these terms, under the hot hands theory, a player whose last three
shots were \texttt{\textquotesingle{}111\textquotesingle{}} (three
consecutive makes) has a higher chance of making a fourth basket than if
their last three shots were
\texttt{\textquotesingle{}110\textquotesingle{}}. The failed third shot
"resets" their hot hands.

The notion of a hot hand is often considered to be a cognitive fallacy,
a tendency for our brains to ascribe more meaning to a random sequence
of shots than it rightly should. People have taken many different
approaches to this topic. This homework shows how one can use
statistical testing tools to test the existence of the hot hands effect
in basketball.

\subsection{The Data}\label{the-data}

Shot records for the Golden State Warriors (our local NBA basketball
team) from the 2016-2017 season are given to you in the
\texttt{data\_dir} path. The files are stored in \texttt{json} format
and are named '\{match\_date\}0\{team\}.json'. \texttt{match\_date} is
the date of the game and \texttt{team} is either 'GSW' or the
abbreviation for the opposing team. The structure of the data is simple:
each file holds shot records for a single game in key/value pairs. The
keys are player names and the values are ordered arrays of shot
attempts. A \texttt{1} represents a "make" (successful attempt) and a
\texttt{0} is a "miss" (failed attempt). Although this will perhaps
overly simplify the analysis, for this assignment, we will not
differentiate between 2-point attempts (2FGA), 3-point attempts (3FGA),
and free-throws (FT).

    \paragraph{Problem 1 {[}5pts{]}}\label{problem-1-5pts}

Write a function \texttt{game\_json\_to\_game\_df} that takes a json
file and builds a dataframe where each row of the table represents the
information about shots for each player. Your table should have three
columns \texttt{player}, \texttt{shots}, and \texttt{game}, described
below:

\begin{itemize}
\tightlist
\item
  \texttt{player}: strings, player name
\item
  \texttt{shots}: strings, the sequence of attempted shots concatenated
  into a single string e.g. '110101'.
\item
  \texttt{game}: strings, the name of the json file (without the .json
  extension)
\end{itemize}

Run the cell below to see an example of the expected output. The index
should just be the numbers 0 through N - 1 (i.e. you don't need to do
anything special to generate the index).

    \begin{Verbatim}[commandchars=\\\{\}]
{\color{incolor}In [{\color{incolor}3}]:} \PY{n}{pd}\PY{o}{.}\PY{n}{read\PYZus{}csv}\PY{p}{(}\PY{l+s+s1}{\PYZsq{}}\PY{l+s+s1}{single\PYZus{}file\PYZus{}shot\PYZus{}data\PYZus{}example.csv}\PY{l+s+s1}{\PYZsq{}}\PY{p}{)}
\end{Verbatim}


\begin{Verbatim}[commandchars=\\\{\}]
{\color{outcolor}Out[{\color{outcolor}3}]:}            player                                 shots          game
        0     A. Iguodala                                   001  201610250GSW
        1      A. Varejao                                    01  201610250GSW
        2      D. Bertans                                    11  201610250GSW
        3       D. Dedmon                                  0010  201610250GSW
        4        D. Green                      0010011110100111  201610250GSW
        5          D. Lee                                110101  201610250GSW
        6         D. West                                    10  201610250GSW
        7        I. Clark                            0011001000  201610250GSW
        8        J. McGee                                   100  201610250GSW
        9      J. Simmons                     11111101001000001  201610250GSW
        10    K. Anderson                                     1  201610250GSW
        11      K. Durant               11110010110001001111111  201610250GSW
        12     K. Leonard  011100111111100101110011001111111110  201610250GSW
        13    K. Thompson                         0000010110101  201610250GSW
        14    L. Aldridge            01101000110111100111111000  201610250GSW
        15    M. Ginobili                            1001000110  201610250GSW
        16       P. Gasol                                  1000  201610250GSW
        17       P. McCaw                                   001  201610250GSW
        18       P. Mills                             001010110  201610250GSW
        19       S. Curry              011111001111100000110110  201610250GSW
        20  S. Livingston                                   010  201610250GSW
        21      T. Parker                             100011001  201610250GSW
        22    Z. Pachulia                                     1  201610250GSW
\end{Verbatim}
            
    \textbf{Hints:}

\begin{enumerate}
\def\labelenumi{\arabic{enumi}.}
\item
  You can load a json file as a dictionary with:

\begin{verbatim}
with open(json_filename) as f:
    data = json.load(f)
\end{verbatim}
\item
  The \texttt{json\_filename} given to you is a
  \href{https://docs.python.org/3/library/pathlib.html}{\texttt{Path\ object}},
  which has a handy method called \texttt{stem} that you might find
  useful.
\end{enumerate}

    \begin{Verbatim}[commandchars=\\\{\}]
{\color{incolor}In [{\color{incolor}4}]:} \PY{k}{def} \PY{n+nf}{game\PYZus{}json\PYZus{}to\PYZus{}game\PYZus{}df}\PY{p}{(}\PY{n}{json\PYZus{}filename}\PY{p}{)}\PY{p}{:}
            \PY{k}{with} \PY{n+nb}{open}\PY{p}{(}\PY{n}{json\PYZus{}filename}\PY{p}{)} \PY{k}{as} \PY{n}{f}\PY{p}{:}
                \PY{n}{data} \PY{o}{=} \PY{n}{json}\PY{o}{.}\PY{n}{load}\PY{p}{(}\PY{n}{f}\PY{p}{)}
            \PY{n}{game} \PY{o}{=} \PY{n}{json\PYZus{}filename}\PY{o}{.}\PY{n}{stem}
            \PY{n}{gamecol} \PY{o}{=} \PY{p}{[}\PY{p}{]}
            \PY{k}{for} \PY{n}{key} \PY{o+ow}{in} \PY{n}{data}\PY{p}{:}
                \PY{n}{result} \PY{o}{=} \PY{l+s+s2}{\PYZdq{}}\PY{l+s+s2}{\PYZdq{}}
                \PY{n}{gamecol} \PY{o}{+}\PY{o}{=} \PY{p}{[}\PY{n}{game}\PY{p}{]}
                \PY{k}{for} \PY{n}{num} \PY{o+ow}{in} \PY{n}{data}\PY{p}{[}\PY{n}{key}\PY{p}{]}\PY{p}{:}
                    \PY{n}{result} \PY{o}{+}\PY{o}{=} \PY{n+nb}{str}\PY{p}{(}\PY{n}{num}\PY{p}{)}
                \PY{n}{data}\PY{p}{[}\PY{n}{key}\PY{p}{]} \PY{o}{=} \PY{n}{result}
            \PY{n}{temp} \PY{o}{=} \PY{n}{pd}\PY{o}{.}\PY{n}{DataFrame}\PY{o}{.}\PY{n}{from\PYZus{}dict}\PY{p}{(}\PY{n}{data}\PY{p}{,} \PY{n}{orient}\PY{o}{=}\PY{l+s+s1}{\PYZsq{}}\PY{l+s+s1}{index}\PY{l+s+s1}{\PYZsq{}}\PY{p}{)}
            \PY{n}{temp}\PY{p}{[}\PY{l+s+s2}{\PYZdq{}}\PY{l+s+s2}{game}\PY{l+s+s2}{\PYZdq{}}\PY{p}{]} \PY{o}{=} \PY{n}{gamecol}
            \PY{n}{temp} \PY{o}{=} \PY{n}{temp}\PY{o}{.}\PY{n}{reset\PYZus{}index}\PY{p}{(}\PY{p}{)}\PY{o}{.}\PY{n}{rename}\PY{p}{(}\PY{n}{columns}\PY{o}{=}\PY{p}{\PYZob{}}\PY{l+s+s2}{\PYZdq{}}\PY{l+s+s2}{index}\PY{l+s+s2}{\PYZdq{}}\PY{p}{:} \PY{l+s+s2}{\PYZdq{}}\PY{l+s+s2}{player}\PY{l+s+s2}{\PYZdq{}}\PY{p}{,} \PY{l+m+mi}{0}\PY{p}{:} \PY{l+s+s2}{\PYZdq{}}\PY{l+s+s2}{shots}\PY{l+s+s2}{\PYZdq{}}\PY{p}{\PYZcb{}}\PY{p}{)}
            \PY{k}{return} \PY{n}{temp}
        \PY{c+c1}{\PYZsh{}raise NotImplementedError()}
\end{Verbatim}


    \begin{Verbatim}[commandchars=\\\{\}]
{\color{incolor}In [{\color{incolor}5}]:} \PY{n}{datafile\PYZus{}path} \PY{o}{=} \PY{n}{Path}\PY{p}{(}\PY{l+s+s1}{\PYZsq{}}\PY{l+s+s1}{data/2017/201610250GSW.json}\PY{l+s+s1}{\PYZsq{}}\PY{p}{)}
        \PY{n}{student\PYZus{}output\PYZus{}201610250GSW} \PY{o}{=} \PY{n}{game\PYZus{}json\PYZus{}to\PYZus{}game\PYZus{}df}\PY{p}{(}\PY{n}{datafile\PYZus{}path}\PY{p}{)}
        \PY{k}{assert} \PY{n}{student\PYZus{}output\PYZus{}201610250GSW}\PY{o}{.}\PY{n}{shape} \PY{o}{==} \PY{p}{(}\PY{l+m+mi}{23}\PY{p}{,} \PY{l+m+mi}{3}\PY{p}{)}\PY{p}{,} \PYZbs{}
        \PY{l+s+s1}{\PYZsq{}}\PY{l+s+s1}{The dimensions of your data frame are incorrect}\PY{l+s+s1}{\PYZsq{}}
        \PY{k}{assert} \PY{l+s+s1}{\PYZsq{}}\PY{l+s+s1}{player}\PY{l+s+s1}{\PYZsq{}} \PY{o+ow}{in} \PY{n}{student\PYZus{}output\PYZus{}201610250GSW}\PY{o}{.}\PY{n}{columns}\PY{o}{.}\PY{n}{values}\PY{p}{,} \PYZbs{}
        \PY{l+s+s1}{\PYZsq{}}\PY{l+s+s1}{You seem to be missing the player column}\PY{l+s+s1}{\PYZsq{}}
        \PY{k}{assert} \PY{l+s+s1}{\PYZsq{}}\PY{l+s+s1}{shots}\PY{l+s+s1}{\PYZsq{}} \PY{o+ow}{in} \PY{n}{student\PYZus{}output\PYZus{}201610250GSW}\PY{o}{.}\PY{n}{columns}\PY{o}{.}\PY{n}{values}\PY{p}{,} \PYZbs{}
        \PY{l+s+s1}{\PYZsq{}}\PY{l+s+s1}{You seem to be missing the shots column}\PY{l+s+s1}{\PYZsq{}}
        \PY{k}{assert} \PY{l+s+s1}{\PYZsq{}}\PY{l+s+s1}{game}\PY{l+s+s1}{\PYZsq{}} \PY{o+ow}{in} \PY{n}{student\PYZus{}output\PYZus{}201610250GSW}\PY{o}{.}\PY{n}{columns}\PY{o}{.}\PY{n}{values}\PY{p}{,} \PYZbs{}
        \PY{l+s+s1}{\PYZsq{}}\PY{l+s+s1}{You seem to be missing the game column}\PY{l+s+s1}{\PYZsq{}}
        \PY{n}{expected\PYZus{}output\PYZus{}201610250GSW} \PY{o}{=} \PY{n}{pd}\PY{o}{.}\PY{n}{read\PYZus{}csv}\PY{p}{(}\PY{l+s+s1}{\PYZsq{}}\PY{l+s+s1}{single\PYZus{}file\PYZus{}shot\PYZus{}data\PYZus{}example.csv}\PY{l+s+s1}{\PYZsq{}}\PY{p}{)}
        \PY{k}{assert}\PY{p}{(}\PY{n}{student\PYZus{}output\PYZus{}201610250GSW}\PY{o}{.}\PY{n}{equals}\PY{p}{(}\PY{n}{expected\PYZus{}output\PYZus{}201610250GSW}\PY{p}{)}\PY{p}{)}
\end{Verbatim}


    \paragraph{Problem 2 {[}5pts{]}}\label{problem-2-5pts}

Read in all 99 \texttt{json} files and combine them into a single data
frame called \texttt{unindexed\_shot\_data}. This dataframe should have
the exact same structure as in the previous part, where the index is
just the numbers 0 through N - 1, where N is the total number of rows in
ALL files. The following cell shows the first 25 rows of the result you
should generate.

\textbf{Hints:}

\begin{enumerate}
\def\labelenumi{\arabic{enumi}.}
\tightlist
\item
  The \texttt{ignore\_index} property of the \texttt{append} method of
  the \texttt{DataFrame} class might be useful.
\item
  The \texttt{glob} method of the \texttt{Path} class might be useful.
\end{enumerate}

    \begin{Verbatim}[commandchars=\\\{\}]
{\color{incolor}In [{\color{incolor}6}]:} \PY{n}{pd}\PY{o}{.}\PY{n}{read\PYZus{}csv}\PY{p}{(}\PY{l+s+s1}{\PYZsq{}}\PY{l+s+s1}{every\PYZus{}file\PYZus{}shot\PYZus{}data\PYZus{}first\PYZus{}25\PYZus{}rows.csv}\PY{l+s+s1}{\PYZsq{}}\PY{p}{)}
\end{Verbatim}


\begin{Verbatim}[commandchars=\\\{\}]
{\color{outcolor}Out[{\color{outcolor}6}]:}            player                                          shots          game
        0     A. Iguodala                                            001  201610250GSW
        1      A. Varejao                                             01  201610250GSW
        2      D. Bertans                                             11  201610250GSW
        3       D. Dedmon                                           0010  201610250GSW
        4        D. Green                               0010011110100111  201610250GSW
        5          D. Lee                                         110101  201610250GSW
        6         D. West                                             10  201610250GSW
        7        I. Clark                                     0011001000  201610250GSW
        8        J. McGee                                            100  201610250GSW
        9      J. Simmons                              11111101001000001  201610250GSW
        10    K. Anderson                                              1  201610250GSW
        11      K. Durant                        11110010110001001111111  201610250GSW
        12     K. Leonard           011100111111100101110011001111111110  201610250GSW
        13    K. Thompson                                  0000010110101  201610250GSW
        14    L. Aldridge                     01101000110111100111111000  201610250GSW
        15    M. Ginobili                                     1001000110  201610250GSW
        16       P. Gasol                                           1000  201610250GSW
        17       P. McCaw                                            001  201610250GSW
        18       P. Mills                                      001010110  201610250GSW
        19       S. Curry                       011111001111100000110110  201610250GSW
        20  S. Livingston                                            010  201610250GSW
        21      T. Parker                                      100011001  201610250GSW
        22    Z. Pachulia                                              1  201610250GSW
        23       A. Davis  111011000011001110110110111100100111100111001  201610280NOP
        24    A. Iguodala                                        0101110  201610280NOP
\end{Verbatim}
            
    \begin{Verbatim}[commandchars=\\\{\}]
{\color{incolor}In [{\color{incolor}7}]:} \PY{n}{data\PYZus{}dir} \PY{o}{=} \PY{n}{Path}\PY{p}{(}\PY{l+s+s1}{\PYZsq{}}\PY{l+s+s1}{data/2017}\PY{l+s+s1}{\PYZsq{}}\PY{p}{)}
        \PY{n}{unindexed\PYZus{}shot\PYZus{}data} \PY{o}{=} \PY{k+kc}{None}
        \PY{k}{for} \PY{n}{jsn} \PY{o+ow}{in} \PY{n}{data\PYZus{}dir}\PY{o}{.}\PY{n}{glob}\PY{p}{(}\PY{l+s+s1}{\PYZsq{}}\PY{l+s+s1}{*.json}\PY{l+s+s1}{\PYZsq{}}\PY{p}{)}\PY{p}{:}
            \PY{k}{if} \PY{n}{unindexed\PYZus{}shot\PYZus{}data} \PY{o+ow}{is} \PY{k+kc}{None}\PY{p}{:}
                \PY{n}{unindexed\PYZus{}shot\PYZus{}data} \PY{o}{=} \PY{n}{game\PYZus{}json\PYZus{}to\PYZus{}game\PYZus{}df}\PY{p}{(}\PY{n}{jsn}\PY{p}{)}
            \PY{k}{else}\PY{p}{:}
                \PY{n}{unindexed\PYZus{}shot\PYZus{}data} \PY{o}{=} \PY{n}{unindexed\PYZus{}shot\PYZus{}data}\PY{o}{.}\PY{n}{append}\PY{p}{(}\PY{n}{game\PYZus{}json\PYZus{}to\PYZus{}game\PYZus{}df}\PY{p}{(}\PY{n}{jsn}\PY{p}{)}\PY{p}{,} \PY{n}{ignore\PYZus{}index}\PY{o}{=}\PY{k+kc}{True}\PY{p}{)}
        
        \PY{c+c1}{\PYZsh{} YOUR CODE HERE}
        \PY{c+c1}{\PYZsh{}raise NotImplementedError()}
\end{Verbatim}


    \begin{Verbatim}[commandchars=\\\{\}]
{\color{incolor}In [{\color{incolor}8}]:} \PY{k}{assert} \PY{n}{unindexed\PYZus{}shot\PYZus{}data}\PY{o}{.}\PY{n}{shape} \PY{o}{==} \PY{p}{(}\PY{l+m+mi}{2144}\PY{p}{,} \PY{l+m+mi}{3}\PY{p}{)}\PY{p}{,} \PYZbs{}
        \PY{l+s+s1}{\PYZsq{}}\PY{l+s+s1}{The dimensions of shot\PYZus{}data are off}\PY{l+s+s1}{\PYZsq{}}
        \PY{k}{assert} \PY{l+s+s1}{\PYZsq{}}\PY{l+s+s1}{shots}\PY{l+s+s1}{\PYZsq{}} \PY{o+ow}{in} \PY{n}{unindexed\PYZus{}shot\PYZus{}data}\PY{o}{.}\PY{n}{columns}\PY{o}{.}\PY{n}{values}\PY{p}{,} \PYZbs{}
        \PY{l+s+s1}{\PYZsq{}}\PY{l+s+s1}{You seem to be missing the shots column}\PY{l+s+s1}{\PYZsq{}}
        \PY{k}{assert} \PY{l+s+s1}{\PYZsq{}}\PY{l+s+s1}{201610250GSW}\PY{l+s+s1}{\PYZsq{}} \PY{o+ow}{in} \PY{n}{unindexed\PYZus{}shot\PYZus{}data}\PY{p}{[}\PY{l+s+s1}{\PYZsq{}}\PY{l+s+s1}{game}\PY{l+s+s1}{\PYZsq{}}\PY{p}{]}\PY{o}{.}\PY{n}{values}\PY{p}{,} \PYZbs{}
        \PY{l+s+s1}{\PYZsq{}}\PY{l+s+s1}{201610280NOP is missing from the game column of the data frame}\PY{l+s+s1}{\PYZsq{}}
        \PY{k}{assert} \PY{l+s+s1}{\PYZsq{}}\PY{l+s+s1}{K. Thompson}\PY{l+s+s1}{\PYZsq{}} \PY{o+ow}{in} \PY{n}{unindexed\PYZus{}shot\PYZus{}data}\PY{p}{[}\PY{l+s+s1}{\PYZsq{}}\PY{l+s+s1}{player}\PY{l+s+s1}{\PYZsq{}}\PY{p}{]}\PY{o}{.}\PY{n}{values}\PY{p}{,} \PYZbs{}
        \PY{l+s+s1}{\PYZsq{}}\PY{l+s+s1}{K. Thompson is missing from the player column of the data frame}\PY{l+s+s1}{\PYZsq{}}
        \PY{k}{assert} \PY{n+nb}{len}\PY{p}{(}\PY{n}{unindexed\PYZus{}shot\PYZus{}data}\PY{p}{[}\PY{l+s+s1}{\PYZsq{}}\PY{l+s+s1}{shots}\PY{l+s+s1}{\PYZsq{}}\PY{p}{]}\PY{o}{.}\PY{n}{values}\PY{o}{.}\PY{n}{sum}\PY{p}{(}\PY{p}{)}\PY{p}{)} \PY{o}{==} \PY{l+m+mi}{22051}\PY{p}{,} \PYZbs{}
        \PY{l+s+s1}{\PYZsq{}}\PY{l+s+s1}{The total number of attempts seems off}\PY{l+s+s1}{\PYZsq{}}
\end{Verbatim}


    Run the line of code below. It converts your integer-indexed data frame
into a multi-indexed one, where the first index is \texttt{game}, and
the second index is \texttt{player}.

    \begin{Verbatim}[commandchars=\\\{\}]
{\color{incolor}In [{\color{incolor}9}]:} \PY{n}{shot\PYZus{}data} \PY{o}{=} \PY{n}{unindexed\PYZus{}shot\PYZus{}data}\PY{o}{.}\PY{n}{set\PYZus{}index}\PY{p}{(}\PY{p}{[}\PY{l+s+s1}{\PYZsq{}}\PY{l+s+s1}{game}\PY{l+s+s1}{\PYZsq{}}\PY{p}{,} \PY{l+s+s1}{\PYZsq{}}\PY{l+s+s1}{player}\PY{l+s+s1}{\PYZsq{}}\PY{p}{]}\PY{p}{)}
        \PY{n}{shot\PYZus{}data}\PY{o}{.}\PY{n}{head}\PY{p}{(}\PY{l+m+mi}{5}\PY{p}{)}
\end{Verbatim}


\begin{Verbatim}[commandchars=\\\{\}]
{\color{outcolor}Out[{\color{outcolor}9}]:}                                        shots
        game         player                         
        201701200HOU A. Iguodala               00101
                     B. Brown                   1110
                     C. Brewer                110010
                     C. Capela    101100111110101101
                     D. Green           100011110001
\end{Verbatim}
            
    \begin{Verbatim}[commandchars=\\\{\}]
{\color{incolor}In [{\color{incolor}10}]:} \PY{k}{assert} \PY{n}{shot\PYZus{}data}\PY{o}{.}\PY{n}{shape} \PY{o}{==} \PY{p}{(}\PY{l+m+mi}{2144}\PY{p}{,} \PY{l+m+mi}{1}\PY{p}{)}\PY{p}{,} \PYZbs{}
         \PY{l+s+s1}{\PYZsq{}}\PY{l+s+s1}{The dimensions of shot\PYZus{}data are off}\PY{l+s+s1}{\PYZsq{}}
         \PY{k}{assert} \PY{l+s+s1}{\PYZsq{}}\PY{l+s+s1}{shots}\PY{l+s+s1}{\PYZsq{}} \PY{o+ow}{in} \PY{n}{shot\PYZus{}data}\PY{o}{.}\PY{n}{columns}\PY{o}{.}\PY{n}{values}\PY{p}{,} \PYZbs{}
         \PY{l+s+s1}{\PYZsq{}}\PY{l+s+s1}{You seem to be missing the shots column}\PY{l+s+s1}{\PYZsq{}}
         \PY{k}{assert} \PY{l+s+s1}{\PYZsq{}}\PY{l+s+s1}{201610250GSW}\PY{l+s+s1}{\PYZsq{}} \PY{o+ow}{in} \PY{n}{shot\PYZus{}data}\PY{o}{.}\PY{n}{index}\PY{o}{.}\PY{n}{get\PYZus{}level\PYZus{}values}\PY{p}{(}\PY{l+m+mi}{0}\PY{p}{)}\PY{p}{,} \PYZbs{}
         \PY{l+s+s1}{\PYZsq{}}\PY{l+s+s1}{201610250GSW is missing from the index}\PY{l+s+s1}{\PYZsq{}}
         \PY{k}{assert} \PY{l+s+s1}{\PYZsq{}}\PY{l+s+s1}{K. Thompson}\PY{l+s+s1}{\PYZsq{}} \PY{o+ow}{in} \PY{n}{shot\PYZus{}data}\PY{o}{.}\PY{n}{index}\PY{o}{.}\PY{n}{get\PYZus{}level\PYZus{}values}\PY{p}{(}\PY{l+m+mi}{1}\PY{p}{)}\PY{p}{,} \PYZbs{}
         \PY{l+s+s1}{\PYZsq{}}\PY{l+s+s1}{K. Thompson is missing from the index}\PY{l+s+s1}{\PYZsq{}}
         \PY{k}{assert} \PY{n+nb}{len}\PY{p}{(}\PY{n}{shot\PYZus{}data}\PY{p}{[}\PY{l+s+s1}{\PYZsq{}}\PY{l+s+s1}{shots}\PY{l+s+s1}{\PYZsq{}}\PY{p}{]}\PY{o}{.}\PY{n}{values}\PY{o}{.}\PY{n}{sum}\PY{p}{(}\PY{p}{)}\PY{p}{)} \PY{o}{==} \PY{l+m+mi}{22051}\PY{p}{,} \PYZbs{}
         \PY{l+s+s1}{\PYZsq{}}\PY{l+s+s1}{The total number of attempts seems off}\PY{l+s+s1}{\PYZsq{}}
\end{Verbatim}


    \subsection{The Hypothesis}\label{the-hypothesis}

Our \textbf{null hypothesis} is that there is no hot hands effect,
meaning that the probability of making shots do not change when a player
makes several baskets in a row. In this null world, every permutation of
a given shot sequence is equally likely. For example
\texttt{\textquotesingle{}00111\textquotesingle{}} is just as likely as
\texttt{\textquotesingle{}10101\textquotesingle{}},
\texttt{\textquotesingle{}10011\textquotesingle{}}, and
\texttt{\textquotesingle{}01101\textquotesingle{}}. In a universe where
hot hands exists, the first sequence would be more likely than the other
three.

Often in modeling the world, we begin by specifying a simplified model
just to see if the question makes sense. We've hidden some other strong
assumptions (perhaps erroneously) about the shots in our model. Here are
some things we are not controlling for:

\begin{itemize}
\tightlist
\item
  Opposing defenders affect the difficulty of a shot
\item
  Distance affects the difficulty of a shot
\item
  Shot types vary in difficulty (3-pointers, 2-points, free-throws)
\item
  Team mate behavior may create more favorable scoring conditions
\end{itemize}

\subsection{Understanding the Data}\label{understanding-the-data}

Recall that as good data scientists, we should strive to understand our
data before we analyze it (data provenance). Let's take a look at
\href{https://www.basketball-reference.com/play-index/shooting.fcgi?player_id=thompkl01\&year_id=2017\&opp_id=IND\&game_location=H}{Klay
Thompson's shooting performance from Dec. 5, 2016 versus the Indiana
Pacers}. Klay scored 60 points in 29 minutes of playing time. For those
of you unfamiliar with basketball, this is a crazy number of points to
score while only being in a game for 30 minutes. In the
\href{https://www.basketball-reference.com/play-index/pgl_finder.cgi?request=1\&match=game\&is_playoffs=N\&age_min=0\&age_max=99\&pos_is_g=Y\&pos_is_gf=Y\&pos_is_f=Y\&pos_is_fg=Y\&pos_is_fc=Y\&pos_is_c=Y\&pos_is_cf=Y\&c1stat=mp\&c1comp=lt\&c1val=30\&order_by=pts}{entire
history of professional basketball}, nobody has come close (note these
records are spotty before 1983).

During this game, Klay took a total of 44 shots, landing 10/11 1 point
free-throws, 13/19 2 point shots, and 8/14 3 point shots.
\href{https://www.usatoday.com/story/sports/nba/warriors/2016/12/06/klay-thompson-60-points-outburst-by-the-numbers-warriors-pacers/95030316/}{At
least one news story} specifically called him out as having a 'hot hand'
during this game.

We'll start by looking at this game to make sure we understanding the
structure of the data.

\paragraph{Problem 3 {[}1pt{]}}\label{problem-3-1pt}

We first summarize Klay's sequence of shot results. Calculate his number
of \texttt{attempts}, number of \texttt{makes} (number of successes,
denoted as 1), and \texttt{accuracy} for this one game. The cell below
stores Klay's shots in the game described above into the klay\_example
variable. Your answer should go in the cell below that.

    \begin{Verbatim}[commandchars=\\\{\}]
{\color{incolor}In [{\color{incolor}11}]:} \PY{n}{klay\PYZus{}example} \PY{o}{=} \PY{n}{shot\PYZus{}data}\PY{o}{.}\PY{n}{loc}\PY{p}{[}\PY{p}{(}\PY{l+s+s1}{\PYZsq{}}\PY{l+s+s1}{201612050GSW}\PY{l+s+s1}{\PYZsq{}}\PY{p}{,} \PY{l+s+s1}{\PYZsq{}}\PY{l+s+s1}{K. Thompson}\PY{l+s+s1}{\PYZsq{}}\PY{p}{)}\PY{p}{,} \PY{l+s+s1}{\PYZsq{}}\PY{l+s+s1}{shots}\PY{l+s+s1}{\PYZsq{}}\PY{p}{]}
         \PY{n}{klay\PYZus{}example}
\end{Verbatim}


\begin{Verbatim}[commandchars=\\\{\}]
{\color{outcolor}Out[{\color{outcolor}11}]:} '11011110010111111001110111101110111101010101'
\end{Verbatim}
            
    \begin{Verbatim}[commandchars=\\\{\}]
{\color{incolor}In [{\color{incolor}12}]:} \PY{n}{attempts\PYZus{}ex} \PY{o}{=} \PY{n+nb}{len}\PY{p}{(}\PY{n}{klay\PYZus{}example}\PY{p}{)}
         \PY{n}{makes\PYZus{}ex} \PY{o}{=} \PY{n+nb}{sum}\PY{p}{(}\PY{p}{[}\PY{n+nb}{int}\PY{p}{(}\PY{n}{i}\PY{p}{)} \PY{k}{for} \PY{n}{i} \PY{o+ow}{in} \PY{n}{klay\PYZus{}example}\PY{p}{]}\PY{p}{)}
         \PY{n}{accuracy\PYZus{}ex} \PY{o}{=} \PY{n}{makes\PYZus{}ex}\PY{o}{/}\PY{n}{attempts\PYZus{}ex}
         
         \PY{c+c1}{\PYZsh{} YOUR CODE HERE}
         \PY{c+c1}{\PYZsh{}raise NotImplementedError()}
         
         \PY{n+nb}{print}\PY{p}{(}\PY{n}{f}\PY{l+s+s2}{\PYZdq{}\PYZdq{}\PYZdq{}}
         \PY{l+s+s2}{attempts:  }\PY{l+s+si}{\PYZob{}attempts\PYZus{}ex\PYZcb{}}
         \PY{l+s+s2}{makes:     }\PY{l+s+si}{\PYZob{}makes\PYZus{}ex\PYZcb{}}
         \PY{l+s+s2}{accuracy:  }\PY{l+s+s2}{\PYZob{}}\PY{l+s+s2}{round(accuracy\PYZus{}ex, 2)\PYZcb{}}
         \PY{l+s+s2}{\PYZdq{}\PYZdq{}\PYZdq{}}\PY{p}{)}
\end{Verbatim}


    \begin{Verbatim}[commandchars=\\\{\}]

attempts:  44
makes:     31
accuracy:  0.7


    \end{Verbatim}

    \begin{Verbatim}[commandchars=\\\{\}]
{\color{incolor}In [{\color{incolor}13}]:} \PY{k}{assert} \PY{n}{attempts\PYZus{}ex} \PY{o}{==} \PY{l+m+mi}{44}
         \PY{k}{assert} \PY{n}{makes\PYZus{}ex} \PY{o}{==} \PY{l+m+mi}{31}
         \PY{k}{assert} \PY{n+nb}{round}\PY{p}{(}\PY{n}{accuracy\PYZus{}ex}\PY{p}{,} \PY{l+m+mi}{2}\PY{p}{)} \PY{o}{==} \PY{l+m+mf}{0.7}
\end{Verbatim}


    We might be interested in the number of runs of various lengths that
Thompson makes over the course of the game. A run of length k is defined
as k consecutive successes in a row. We will include overlapping runs in
our counts. For example, the shot record
\texttt{\textquotesingle{}1111\textquotesingle{}} contains three runs of
length 2: {11}11, 1{11}1, 11{11}).

\paragraph{Problem 4 {[}2pts{]}}\label{problem-4-2pts}

How many runs of length 2 did Thompson make in the Dec. 5, 2016 game? To
answer this question, we used a regular expression, but you're free to
answer this however you'd like (with code, of course). In our regular
expression we make use of
\href{https://docs.python.org/2/library/re.html}{positive lookbehinds}
\texttt{(?\textless{}=...)}.

    \begin{Verbatim}[commandchars=\\\{\}]
{\color{incolor}In [{\color{incolor}14}]:} \PY{n}{run\PYZus{}length\PYZus{}2} \PY{o}{=} \PY{n+nb}{len}\PY{p}{(}\PY{n}{re}\PY{o}{.}\PY{n}{findall}\PY{p}{(}\PY{l+s+s1}{\PYZsq{}}\PY{l+s+s1}{(?\PYZlt{}=1)1}\PY{l+s+s1}{\PYZsq{}}\PY{p}{,} \PY{n}{klay\PYZus{}example}\PY{p}{)}\PY{p}{)}
         
         \PY{c+c1}{\PYZsh{} YOUR CODE HERE}
         \PY{c+c1}{\PYZsh{}raise NotImplementedError()}
         
         \PY{n+nb}{print}\PY{p}{(}\PY{n}{f}\PY{l+s+s2}{\PYZdq{}\PYZdq{}\PYZdq{}}
         \PY{l+s+s2}{Klay Thompson made }\PY{l+s+si}{\PYZob{}run\PYZus{}length\PYZus{}2\PYZcb{}}\PY{l+s+s2}{ runs of length 2 in the game against the Indiana Pacers.}
         \PY{l+s+s2}{\PYZdq{}\PYZdq{}\PYZdq{}}\PY{p}{)}
\end{Verbatim}


    \begin{Verbatim}[commandchars=\\\{\}]

Klay Thompson made 19 runs of length 2 in the game against the Indiana Pacers.


    \end{Verbatim}

    \begin{Verbatim}[commandchars=\\\{\}]
{\color{incolor}In [{\color{incolor}15}]:} \PY{k}{assert} \PY{n}{run\PYZus{}length\PYZus{}2} \PY{o}{==} \PY{l+m+mi}{19}
\end{Verbatim}


    \paragraph{Problem 5 {[}2pts{]}}\label{problem-5-2pts}

How many runs of length 3?

    \begin{Verbatim}[commandchars=\\\{\}]
{\color{incolor}In [{\color{incolor}16}]:} \PY{n}{run\PYZus{}length\PYZus{}3} \PY{o}{=} \PY{n+nb}{len}\PY{p}{(}\PY{n}{re}\PY{o}{.}\PY{n}{findall}\PY{p}{(}\PY{l+s+s1}{\PYZsq{}}\PY{l+s+s1}{(?\PYZlt{}=11)1}\PY{l+s+s1}{\PYZsq{}}\PY{p}{,} \PY{n}{klay\PYZus{}example}\PY{p}{)}\PY{p}{)}
         
         \PY{c+c1}{\PYZsh{} YOUR CODE HERE}
         \PY{c+c1}{\PYZsh{}raise NotImplementedError()}
         
         \PY{n+nb}{print}\PY{p}{(}\PY{n}{f}\PY{l+s+s2}{\PYZdq{}\PYZdq{}\PYZdq{}}
         \PY{l+s+s2}{Klay Thompson made }\PY{l+s+si}{\PYZob{}run\PYZus{}length\PYZus{}3\PYZcb{}}\PY{l+s+s2}{ runs of length 3 in the game against the Indiana Pacers.}
         \PY{l+s+s2}{\PYZdq{}\PYZdq{}\PYZdq{}}\PY{p}{)}
\end{Verbatim}


    \begin{Verbatim}[commandchars=\\\{\}]

Klay Thompson made 12 runs of length 3 in the game against the Indiana Pacers.


    \end{Verbatim}

    \begin{Verbatim}[commandchars=\\\{\}]
{\color{incolor}In [{\color{incolor}17}]:} \PY{c+c1}{\PYZsh{} Empty, soulless cells like these contain hidden tests}
         \PY{c+c1}{\PYZsh{} Do not delete}
\end{Verbatim}


    \paragraph{Problem 6 {[}10pts{]}}\label{problem-6-10pts}

Let's generalize the work we did above by writing a function
\texttt{count\_runs}. \texttt{count\_runs} takes two arguments:

\begin{itemize}
\tightlist
\item
  \texttt{shot\_sequences}: a pandas series of strings, each
  representing a sequence of shots for a player in a game
\item
  \texttt{run\_length}: integer, the run length to count
\end{itemize}

\texttt{count\_runs} should return a pandas series, where the ith
element is the number of occurrences of \texttt{run\_length} in the ith
sequence in \texttt{shot\_sequences}.

Some example input/outputs for \texttt{count\_runs} are given below:

\begin{itemize}
\tightlist
\item
  \texttt{count\_runs(pd.Series({[}\textquotesingle{}111\textquotesingle{},\ \textquotesingle{}000\textquotesingle{},\ \textquotesingle{}011\textquotesingle{},\ \textquotesingle{}000\textquotesingle{}{]}),\ 2)}
  should return \texttt{pd.Series({[}2,\ 0,\ 1,\ 0{]})}
\item
  \texttt{count\_runs(pd.Series({[}\textquotesingle{}1100110011\textquotesingle{}{]}),\ 2)}
  should return \texttt{pd.Series({[}3{]})}
\end{itemize}

For convenience, \texttt{count\_runs} should also work if
\texttt{shot\_sequences} is a single string representing a single game,
e.g.

\texttt{count\_runs((1100110011),\ 2)} should return
\texttt{pd.Series({[}3{]})}

    \begin{Verbatim}[commandchars=\\\{\}]
{\color{incolor}In [{\color{incolor}18}]:} \PY{k}{def} \PY{n+nf}{count\PYZus{}runs}\PY{p}{(}\PY{n}{shot\PYZus{}sequences}\PY{p}{,} \PY{n}{run\PYZus{}length}\PY{p}{)}\PY{p}{:}
             \PY{l+s+sd}{\PYZdq{}\PYZdq{}\PYZdq{}}
         \PY{l+s+sd}{    Counts consecutive occurences of an event}
         \PY{l+s+sd}{    }
         \PY{l+s+sd}{    shot\PYZus{}sequences: a pandas series of strings, each representing a sequence of shots for a player in a game}
         \PY{l+s+sd}{    run\PYZus{}length: integer, the run length to count}
         \PY{l+s+sd}{    }
         \PY{l+s+sd}{    return: pd.Series of the number of times a run of length run\PYZus{}length occurred in each shot\PYZus{}sequence}
         \PY{l+s+sd}{    \PYZdq{}\PYZdq{}\PYZdq{}}
             \PY{n}{res}\PY{o}{=}\PY{p}{[}\PY{p}{]}
             \PY{n}{one}\PY{o}{=}\PY{l+s+s2}{\PYZdq{}}\PY{l+s+s2}{\PYZdq{}}
             \PY{k}{for} \PY{n}{i} \PY{o+ow}{in} \PY{n}{np}\PY{o}{.}\PY{n}{arange}\PY{p}{(}\PY{n}{run\PYZus{}length}\PY{o}{\PYZhy{}}\PY{l+m+mi}{1}\PY{p}{)}\PY{p}{:}
                 \PY{n}{one} \PY{o}{+}\PY{o}{=} \PY{l+s+s2}{\PYZdq{}}\PY{l+s+s2}{1}\PY{l+s+s2}{\PYZdq{}}
             \PY{k}{if} \PY{n+nb}{type}\PY{p}{(}\PY{n}{shot\PYZus{}sequences}\PY{p}{)} \PY{o}{==} \PY{n+nb}{str}\PY{p}{:}
                 \PY{n}{shot\PYZus{}sequences} \PY{o}{=} \PY{n}{pd}\PY{o}{.}\PY{n}{Series}\PY{p}{(}\PY{n}{shot\PYZus{}sequences}\PY{p}{)}
             \PY{k}{for} \PY{n}{shot} \PY{o+ow}{in} \PY{n}{pd}\PY{o}{.}\PY{n}{Series}\PY{p}{(}\PY{n}{shot\PYZus{}sequences}\PY{p}{)}\PY{p}{:}
                 \PY{n}{res} \PY{o}{+}\PY{o}{=} \PY{p}{[}\PY{n+nb}{len}\PY{p}{(}\PY{n}{re}\PY{o}{.}\PY{n}{findall}\PY{p}{(}\PY{l+s+s1}{\PYZsq{}}\PY{l+s+s1}{(?\PYZlt{}=}\PY{l+s+s1}{\PYZsq{}}\PY{o}{+}\PY{n}{one}\PY{o}{+}\PY{l+s+s1}{\PYZsq{}}\PY{l+s+s1}{)1}\PY{l+s+s1}{\PYZsq{}}\PY{p}{,} \PY{n}{shot}\PY{p}{)}\PY{p}{)}\PY{p}{]}
             \PY{k}{return} \PY{n}{pd}\PY{o}{.}\PY{n}{Series}\PY{p}{(}\PY{n}{res}\PY{p}{)}
             \PY{c+c1}{\PYZsh{} YOUR CODE HERE}
\end{Verbatim}


    \begin{Verbatim}[commandchars=\\\{\}]
{\color{incolor}In [{\color{incolor}19}]:} \PY{k}{assert} \PY{n}{count\PYZus{}runs}\PY{p}{(}\PY{n}{pd}\PY{o}{.}\PY{n}{Series}\PY{p}{(}\PY{p}{[}\PY{l+s+s1}{\PYZsq{}}\PY{l+s+s1}{111}\PY{l+s+s1}{\PYZsq{}}\PY{p}{,} \PY{l+s+s1}{\PYZsq{}}\PY{l+s+s1}{000}\PY{l+s+s1}{\PYZsq{}}\PY{p}{,} \PY{l+s+s1}{\PYZsq{}}\PY{l+s+s1}{011}\PY{l+s+s1}{\PYZsq{}}\PY{p}{,} \PY{l+s+s1}{\PYZsq{}}\PY{l+s+s1}{000}\PY{l+s+s1}{\PYZsq{}}\PY{p}{]}\PY{p}{)}\PY{p}{,} \PY{l+m+mi}{2}\PY{p}{)}\PY{o}{.}\PY{n}{equals}\PY{p}{(}\PY{n}{pd}\PY{o}{.}\PY{n}{Series}\PY{p}{(}\PY{p}{[}\PY{l+m+mi}{2}\PY{p}{,} \PY{l+m+mi}{0}\PY{p}{,} \PY{l+m+mi}{1}\PY{p}{,} \PY{l+m+mi}{0}\PY{p}{]}\PY{p}{)}\PY{p}{)}\PY{p}{,} \PYZbs{}
         \PY{l+s+s1}{\PYZsq{}}\PY{l+s+s1}{There should be 2, 0, 1, and 0 runs of length 2, respectively.}\PY{l+s+s1}{\PYZsq{}}
         \PY{k}{assert} \PY{n}{count\PYZus{}runs}\PY{p}{(}\PY{n}{pd}\PY{o}{.}\PY{n}{Series}\PY{p}{(}\PY{p}{[}\PY{l+s+s1}{\PYZsq{}}\PY{l+s+s1}{1100110011}\PY{l+s+s1}{\PYZsq{}}\PY{p}{]}\PY{p}{)}\PY{p}{,} \PY{l+m+mi}{2}\PY{p}{)}\PY{o}{.}\PY{n}{equals}\PY{p}{(}\PY{n}{pd}\PY{o}{.}\PY{n}{Series}\PY{p}{(}\PY{p}{[}\PY{l+m+mi}{3}\PY{p}{]}\PY{p}{)}\PY{p}{)}\PY{p}{,} \PYZbs{}
         \PY{l+s+s1}{\PYZsq{}}\PY{l+s+s1}{There should be 1 run of length 3}\PY{l+s+s1}{\PYZsq{}}
         \PY{k}{assert} \PY{n}{count\PYZus{}runs}\PY{p}{(}\PY{l+s+s1}{\PYZsq{}}\PY{l+s+s1}{000}\PY{l+s+s1}{\PYZsq{}}\PY{p}{,} \PY{l+m+mi}{1}\PY{p}{)}\PY{o}{.}\PY{n}{equals}\PY{p}{(}\PY{n}{pd}\PY{o}{.}\PY{n}{Series}\PY{p}{(}\PY{l+m+mi}{0}\PY{p}{)}\PY{p}{)}\PY{p}{,} \PYZbs{}
         \PY{l+s+s1}{\PYZsq{}}\PY{l+s+s1}{There should be 0 runs of 1, and your code must support string inputs (hint: if the input a string, convert it into a pandas series)}\PY{l+s+s1}{\PYZsq{}}
\end{Verbatim}


    \begin{Verbatim}[commandchars=\\\{\}]
{\color{incolor}In [{\color{incolor}20}]:} \PY{c+c1}{\PYZsh{} *Leers*}
\end{Verbatim}


    \paragraph{Problem 7 {[}5pts{]}}\label{problem-7-5pts}

Use \texttt{count\_runs} to transform the data as follows: for each
player, count the number of times they have made a run of length \(k\)
where \(k = 1, 2, 3, ..., 10\). The column names should be
\texttt{str(k)} and the index be the player names. A sample of the
output is given below for three players in the data. The count should be
across all games played by the player across the entire dataset.

    \begin{Verbatim}[commandchars=\\\{\}]
{\color{incolor}In [{\color{incolor}21}]:} \PY{n}{pd}\PY{o}{.}\PY{n}{read\PYZus{}csv}\PY{p}{(}\PY{l+s+s1}{\PYZsq{}}\PY{l+s+s1}{count\PYZus{}runs\PYZus{}example.csv}\PY{l+s+s1}{\PYZsq{}}\PY{p}{,} \PY{n}{index\PYZus{}col}\PY{o}{=}\PY{l+s+s1}{\PYZsq{}}\PY{l+s+s1}{player}\PY{l+s+s1}{\PYZsq{}}\PY{p}{)}
\end{Verbatim}


\begin{Verbatim}[commandchars=\\\{\}]
{\color{outcolor}Out[{\color{outcolor}21}]:}                 1    2    3    4    5   6   7   8   9  10
         player                                                   
         K. Thompson   950  491  251  126   62  31  13   4   1   0
         S. Curry     1269  714  392  200   94  41  14   5   2   1
         K. Durant    1128  695  410  243  136  80  44  24  14   7
\end{Verbatim}
            
    \begin{Verbatim}[commandchars=\\\{\}]
{\color{incolor}In [{\color{incolor}22}]:} \PY{n}{run\PYZus{}counts} \PY{o}{=} \PY{n}{shot\PYZus{}data}\PY{o}{.}\PY{n}{reset\PYZus{}index}\PY{p}{(}\PY{p}{)}\PY{o}{.}\PY{n}{copy}\PY{p}{(}\PY{p}{)}
         \PY{k}{for} \PY{n}{i} \PY{o+ow}{in} \PY{n}{np}\PY{o}{.}\PY{n}{arange}\PY{p}{(}\PY{l+m+mi}{10}\PY{p}{)}\PY{o}{+}\PY{l+m+mi}{1}\PY{p}{:}
             \PY{n}{run\PYZus{}counts}\PY{p}{[}\PY{n+nb}{str}\PY{p}{(}\PY{n}{i}\PY{p}{)}\PY{p}{]} \PY{o}{=} \PY{n}{count\PYZus{}runs}\PY{p}{(}\PY{n}{run\PYZus{}counts}\PY{p}{[}\PY{l+s+s2}{\PYZdq{}}\PY{l+s+s2}{shots}\PY{l+s+s2}{\PYZdq{}}\PY{p}{]}\PY{p}{,} \PY{n}{i}\PY{p}{)}
         \PY{n}{run\PYZus{}counts} \PY{o}{=} \PY{n}{run\PYZus{}counts}\PY{o}{.}\PY{n}{set\PYZus{}index}\PY{p}{(}\PY{l+s+s2}{\PYZdq{}}\PY{l+s+s2}{player}\PY{l+s+s2}{\PYZdq{}}\PY{p}{)}\PY{o}{.}\PY{n}{drop}\PY{p}{(}\PY{p}{[}\PY{l+s+s2}{\PYZdq{}}\PY{l+s+s2}{shots}\PY{l+s+s2}{\PYZdq{}}\PY{p}{,} \PY{l+s+s2}{\PYZdq{}}\PY{l+s+s2}{game}\PY{l+s+s2}{\PYZdq{}}\PY{p}{]}\PY{p}{,} \PY{n}{axis}\PY{o}{=}\PY{l+m+mi}{1}\PY{p}{)}\PY{o}{.}\PY{n}{groupby}\PY{p}{(}\PY{l+s+s2}{\PYZdq{}}\PY{l+s+s2}{player}\PY{l+s+s2}{\PYZdq{}}\PY{p}{)}\PY{o}{.}\PY{n}{sum}\PY{p}{(}\PY{p}{)}
         \PY{c+c1}{\PYZsh{} YOUR CODE HERE}
         \PY{c+c1}{\PYZsh{}raise NotImplementedError()}
\end{Verbatim}


    \begin{Verbatim}[commandchars=\\\{\}]
{\color{incolor}In [{\color{incolor}23}]:} \PY{k}{assert} \PY{n}{pd}\PY{o}{.}\PY{n}{api}\PY{o}{.}\PY{n}{types}\PY{o}{.}\PY{n}{is\PYZus{}string\PYZus{}dtype}\PY{p}{(}\PY{n}{run\PYZus{}counts}\PY{o}{.}\PY{n}{index}\PY{p}{)}\PY{p}{,} \PYZbs{}
         \PY{l+s+s1}{\PYZsq{}}\PY{l+s+s1}{Index should consist of strings.}\PY{l+s+s1}{\PYZsq{}}
         \PY{k}{assert} \PY{n}{pd}\PY{o}{.}\PY{n}{api}\PY{o}{.}\PY{n}{types}\PY{o}{.}\PY{n}{is\PYZus{}string\PYZus{}dtype}\PY{p}{(}\PY{n}{run\PYZus{}counts}\PY{o}{.}\PY{n}{columns}\PY{p}{)}\PY{p}{,} \PYZbs{}
         \PY{l+s+s1}{\PYZsq{}}\PY{l+s+s1}{Column names should be strings.}\PY{l+s+s1}{\PYZsq{}}
         \PY{k}{assert} \PY{n}{run\PYZus{}counts}\PY{o}{.}\PY{n}{loc}\PY{p}{[}\PY{l+s+s1}{\PYZsq{}}\PY{l+s+s1}{A. Abrines}\PY{l+s+s1}{\PYZsq{}}\PY{p}{,} \PY{l+s+s1}{\PYZsq{}}\PY{l+s+s1}{1}\PY{l+s+s1}{\PYZsq{}}\PY{p}{]} \PY{o}{==} \PY{l+m+mi}{8}\PY{p}{,} \PYZbs{}
         \PY{l+s+s1}{\PYZsq{}}\PY{l+s+s1}{A. Abrines should have 8 single makes.}\PY{l+s+s1}{\PYZsq{}}
         \PY{k}{assert} \PY{n}{run\PYZus{}counts}\PY{o}{.}\PY{n}{loc}\PY{p}{[}\PY{l+s+s1}{\PYZsq{}}\PY{l+s+s1}{K. Thompson}\PY{l+s+s1}{\PYZsq{}}\PY{p}{]}\PY{o}{.}\PY{n}{sum}\PY{p}{(}\PY{p}{)} \PY{o}{==} \PY{l+m+mi}{1929}\PY{p}{,} \PYZbs{}
         \PY{l+s+s2}{\PYZdq{}}\PY{l+s+s2}{The sum of K Thompson}\PY{l+s+s2}{\PYZsq{}}\PY{l+s+s2}{s values seems off.}\PY{l+s+s2}{\PYZdq{}}
\end{Verbatim}


    So far, we've just been exploring the data. The \texttt{run\_counts}
table you built above does not provide us any sort of information about
the validity of the hot hands hypothesis.

\texttt{run\_counts} does seem to indicate that very long streaks are
pretty rare. We'll use this as a starting point for our analysis in the
next section.

    \subsection{Defining a Test Statistic}\label{defining-a-test-statistic}

People who refer to "hot hands" often treat it as Justice Potter Stewart
treats obscenity:
\href{https://en.wikipedia.org/wiki/I_know_it_when_I_see_it}{"I know it
when I see it."} As data scientists, this isn't good enough for us.
Instead, we should think about how to quantify the question in an
empirically verifiable way.

Unfortunately, it's not immediately clear how we might test the null
hypothesis. In other hypothesis test settings like website A/B testing
and drug efficacy, we have obvious choices for important and measurable
outcomes to demonstrate increases in revenue or positive health impacts,
respectively.

However, the hot hands is not as well-defined, so we're going to try a
few things that seem to have the flavor of measuring "streakiness".

\paragraph{Problem 8 {[}10pts{]}}\label{problem-8-10pts}

Our first attempt at a test statistic will be the length of the longest
streak. We saw in the previous section that long runs were rare, so
perhaps we can use the occurrence of long runs as evidence either for or
against the hot hands hypothesis.

Write a function \texttt{find\_longest\_run} that computes this test
statistics. Specifically, \texttt{find\_longest\_run} should takes a
\texttt{pd.Series} of shot sequences and returns a \texttt{pd.Series} of
the lengths of the longest make sequences (consecutive 1s) in each
sequence. As with \texttt{run\_counts}, for convenience, make the
function work for a python string input as well.

For example: *
\texttt{find\_longest\_run(pd.Series({[}\textquotesingle{}111\textquotesingle{},\ \textquotesingle{}000\textquotesingle{},\ \textquotesingle{}011\textquotesingle{},\ \textquotesingle{}000\textquotesingle{}{]}))}
should return \texttt{pd.Series({[}3,\ 0,\ 2,\ 0{]})} *
\texttt{find\_longest\_run(pd.Series({[}\textquotesingle{}1100110011\textquotesingle{}{]}))}
should return \texttt{pd.Series({[}2{]})} *
\texttt{find\_longest\_run(\textquotesingle{}1100110011\textquotesingle{})}
should return \texttt{pd.Series({[}2{]})}

    \begin{Verbatim}[commandchars=\\\{\}]
{\color{incolor}In [{\color{incolor}24}]:} \PY{k}{def} \PY{n+nf}{find\PYZus{}longest\PYZus{}run}\PY{p}{(}\PY{n}{shot\PYZus{}sequences}\PY{p}{)}\PY{p}{:}
             \PY{l+s+sd}{\PYZdq{}\PYZdq{}\PYZdq{}}
         \PY{l+s+sd}{    Finds longest run in a pd.Series of shot\PYZus{}sequences}
         \PY{l+s+sd}{    }
         \PY{l+s+sd}{    shot\PYZus{}sequences: pd.Series (string) shot data for a set of games or a single python string}
         \PY{l+s+sd}{       to be coerced into a pd.Series}
         \PY{l+s+sd}{    }
         \PY{l+s+sd}{    return: as pd.Series of the lengths of longest sequences of 1s in each game}
         \PY{l+s+sd}{    \PYZdq{}\PYZdq{}\PYZdq{}}
             \PY{c+c1}{\PYZsh{} YOUR CODE HERE}
             \PY{n}{res}\PY{o}{=}\PY{p}{[}\PY{p}{]}
             \PY{k}{if} \PY{n+nb}{type}\PY{p}{(}\PY{n}{shot\PYZus{}sequences}\PY{p}{)} \PY{o}{==} \PY{n+nb}{str}\PY{p}{:}
                 \PY{n}{shot\PYZus{}sequences} \PY{o}{=} \PY{n}{pd}\PY{o}{.}\PY{n}{Series}\PY{p}{(}\PY{n}{shot\PYZus{}sequences}\PY{p}{)}
             \PY{k}{for} \PY{n}{shot} \PY{o+ow}{in} \PY{n}{pd}\PY{o}{.}\PY{n}{Series}\PY{p}{(}\PY{n}{shot\PYZus{}sequences}\PY{p}{)}\PY{p}{:}
                 \PY{n}{i} \PY{o}{=} \PY{l+m+mi}{1}
                 \PY{k}{while} \PY{n}{count\PYZus{}runs}\PY{p}{(}\PY{n}{shot}\PY{p}{,} \PY{n}{i}\PY{p}{)}\PY{p}{[}\PY{l+m+mi}{0}\PY{p}{]} \PY{o}{\PYZgt{}} \PY{l+m+mi}{0}\PY{p}{:}
                     \PY{n}{i}\PY{o}{+}\PY{o}{=}\PY{l+m+mi}{1}
                 \PY{n}{res}\PY{o}{+}\PY{o}{=}\PY{p}{[}\PY{n}{i}\PY{o}{\PYZhy{}}\PY{l+m+mi}{1}\PY{p}{]}
             \PY{k}{return} \PY{n}{pd}\PY{o}{.}\PY{n}{Series}\PY{p}{(}\PY{n}{res}\PY{p}{)}
         \PY{c+c1}{\PYZsh{}raise NotImplementedError()}
\end{Verbatim}


    \begin{Verbatim}[commandchars=\\\{\}]
{\color{incolor}In [{\color{incolor}25}]:} \PY{k}{assert} \PY{n+nb}{isinstance}\PY{p}{(}\PY{n}{find\PYZus{}longest\PYZus{}run}\PY{p}{(}\PY{n}{klay\PYZus{}example}\PY{p}{)}\PY{p}{,} \PY{n}{pd}\PY{o}{.}\PY{n}{Series}\PY{p}{)}\PY{p}{,} \PYZbs{}
         \PY{l+s+s1}{\PYZsq{}}\PY{l+s+s1}{The output should be a pd.Series}\PY{l+s+s1}{\PYZsq{}}
         \PY{k}{assert} \PY{n}{find\PYZus{}longest\PYZus{}run}\PY{p}{(}\PY{n}{pd}\PY{o}{.}\PY{n}{Series}\PY{p}{(}\PY{p}{[}\PY{l+s+s1}{\PYZsq{}}\PY{l+s+s1}{111}\PY{l+s+s1}{\PYZsq{}}\PY{p}{,} \PY{l+s+s1}{\PYZsq{}}\PY{l+s+s1}{000}\PY{l+s+s1}{\PYZsq{}}\PY{p}{,} \PY{l+s+s1}{\PYZsq{}}\PY{l+s+s1}{011}\PY{l+s+s1}{\PYZsq{}}\PY{p}{,} \PY{l+s+s1}{\PYZsq{}}\PY{l+s+s1}{000}\PY{l+s+s1}{\PYZsq{}}\PY{p}{]}\PY{p}{)}\PY{p}{)}\PY{o}{.}\PY{n}{equals}\PY{p}{(}\PY{n}{pd}\PY{o}{.}\PY{n}{Series}\PY{p}{(}\PY{p}{[}\PY{l+m+mi}{3}\PY{p}{,} \PY{l+m+mi}{0}\PY{p}{,} \PY{l+m+mi}{2}\PY{p}{,} \PY{l+m+mi}{0}\PY{p}{]}\PY{p}{)}\PY{p}{)}\PY{p}{,} \PYZbs{}
         \PY{l+s+s1}{\PYZsq{}}\PY{l+s+s1}{The longest runs should be of length 3, 0, 2, and 0, respectively.}\PY{l+s+s1}{\PYZsq{}}
         \PY{k}{assert} \PY{n}{find\PYZus{}longest\PYZus{}run}\PY{p}{(}\PY{n}{pd}\PY{o}{.}\PY{n}{Series}\PY{p}{(}\PY{p}{[}\PY{l+s+s1}{\PYZsq{}}\PY{l+s+s1}{1100110011}\PY{l+s+s1}{\PYZsq{}}\PY{p}{]}\PY{p}{)}\PY{p}{)}\PY{o}{.}\PY{n}{equals}\PY{p}{(}\PY{n}{pd}\PY{o}{.}\PY{n}{Series}\PY{p}{(}\PY{p}{[}\PY{l+m+mi}{2}\PY{p}{]}\PY{p}{)}\PY{p}{)}\PY{p}{,} \PYZbs{}
         \PY{l+s+s1}{\PYZsq{}}\PY{l+s+s1}{The longest run should be of length 2.}\PY{l+s+s1}{\PYZsq{}}
\end{Verbatim}


    \begin{Verbatim}[commandchars=\\\{\}]
{\color{incolor}In [{\color{incolor}26}]:} \PY{c+c1}{\PYZsh{} Nothing to see here.  Move along}
\end{Verbatim}


    \paragraph{Problem 9 {[}10pts{]}}\label{problem-9-10pts}

If you look at the test inputs above, you'll see that the extreme game
featuring Klay Thompson scoring 60 points in 29 minutes has a longest
run length of 6.

Let's try to understand whether this value for our test statistic is
indicative of Klay having a hot hand during this game. To do this, we
need to know how 6 stacks up as a streak compared to a player similar to
Klay but who definitely does not have a hot hand effect.

How do we find data on such a player? Well, \textbf{under the null
hypothesis, Klay \emph{himself} is such a player}, and the shot record
we observe is really a sequence of independent shots. This suggests a
bootstrap procedure to estimate the sampling distribution of longest
runs. Write a function called \texttt{bootstrap\_longest\_run} that
simulates the sampling distribution of the \texttt{longest\_run} test
statistic under the null hypothesis given the shot record of a single
game. For example, \texttt{bootstrap\_longest\_run(klay\_example,\ 100)}
should return a pandas series of longest runs for 100 simulated games,
where the simulated games are bootstrapped from the Klay example.

    \begin{Verbatim}[commandchars=\\\{\}]
{\color{incolor}In [{\color{incolor}27}]:} \PY{k}{def} \PY{n+nf}{bootstrap\PYZus{}longest\PYZus{}run}\PY{p}{(}\PY{n}{game}\PY{p}{,} \PY{n}{num\PYZus{}iter}\PY{o}{=}\PY{l+m+mi}{1}\PY{p}{)}\PY{p}{:}
             \PY{l+s+sd}{\PYZdq{}\PYZdq{}\PYZdq{}}
         \PY{l+s+sd}{    game: string, shot sequence data for a single game}
         \PY{l+s+sd}{    num\PYZus{}iter: number of statistics to generate}
         \PY{l+s+sd}{    }
         \PY{l+s+sd}{    returns: num\PYZus{}iter statistics drawn from the bootstrapped sampling distribution}
         \PY{l+s+sd}{    \PYZdq{}\PYZdq{}\PYZdq{}}
             \PY{n}{attempts} \PY{o}{=} \PY{n+nb}{len}\PY{p}{(}\PY{n}{game}\PY{p}{)}
             \PY{n}{makes} \PY{o}{=} \PY{n+nb}{sum}\PY{p}{(}\PY{p}{[}\PY{n+nb}{int}\PY{p}{(}\PY{n}{i}\PY{p}{)} \PY{k}{for} \PY{n}{i} \PY{o+ow}{in} \PY{n}{game}\PY{p}{]}\PY{p}{)}
             \PY{n}{accuracy} \PY{o}{=} \PY{n}{makes}\PY{o}{/}\PY{n}{attempts}
             \PY{n}{res}\PY{o}{=}\PY{p}{[}\PY{p}{]}
             \PY{k}{for} \PY{n}{\PYZus{}} \PY{o+ow}{in} \PY{n}{np}\PY{o}{.}\PY{n}{arange}\PY{p}{(}\PY{n}{num\PYZus{}iter}\PY{p}{)}\PY{p}{:}
                 \PY{n}{inpt} \PY{o}{=} \PY{l+s+s2}{\PYZdq{}}\PY{l+s+s2}{\PYZdq{}}
                 \PY{k}{for} \PY{n}{\PYZus{}\PYZus{}} \PY{o+ow}{in} \PY{n}{np}\PY{o}{.}\PY{n}{arange}\PY{p}{(}\PY{n+nb}{len}\PY{p}{(}\PY{n}{game}\PY{p}{)}\PY{p}{)}\PY{p}{:}
                     \PY{k}{if} \PY{n}{np}\PY{o}{.}\PY{n}{random}\PY{o}{.}\PY{n}{uniform}\PY{p}{(}\PY{p}{)} \PY{o}{\PYZlt{}} \PY{n}{accuracy}\PY{p}{:}
                         \PY{n}{inpt} \PY{o}{+}\PY{o}{=} \PY{l+s+s2}{\PYZdq{}}\PY{l+s+s2}{1}\PY{l+s+s2}{\PYZdq{}}
                     \PY{k}{else}\PY{p}{:}
                         \PY{n}{inpt} \PY{o}{+}\PY{o}{=} \PY{l+s+s2}{\PYZdq{}}\PY{l+s+s2}{0}\PY{l+s+s2}{\PYZdq{}}
                 \PY{n}{res} \PY{o}{+}\PY{o}{=} \PY{p}{[}\PY{n}{inpt}\PY{p}{]}
             \PY{k}{return} \PY{n}{find\PYZus{}longest\PYZus{}run}\PY{p}{(}\PY{n}{res}\PY{p}{)}
             \PY{c+c1}{\PYZsh{} YOUR CODE HERE}
             \PY{c+c1}{\PYZsh{}raise NotImplementedError()}
\end{Verbatim}


    \begin{Verbatim}[commandchars=\\\{\}]
{\color{incolor}In [{\color{incolor}28}]:} \PY{n}{longest\PYZus{}run\PYZus{}simulations} \PY{o}{=} \PY{n}{bootstrap\PYZus{}longest\PYZus{}run}\PY{p}{(}\PY{n}{klay\PYZus{}example}\PY{p}{,} \PY{l+m+mi}{100}\PY{p}{)}
         \PY{k}{assert} \PY{n+nb}{isinstance}\PY{p}{(}\PY{n}{longest\PYZus{}run\PYZus{}simulations}\PY{p}{,} \PY{n}{pd}\PY{o}{.}\PY{n}{Series}\PY{p}{)}
         \PY{k}{assert} \PY{n+nb}{len}\PY{p}{(}\PY{n}{longest\PYZus{}run\PYZus{}simulations}\PY{p}{)} \PY{o}{==} \PY{l+m+mi}{100}
         \PY{k}{assert} \PY{n}{longest\PYZus{}run\PYZus{}simulations}\PY{o}{.}\PY{n}{max}\PY{p}{(}\PY{p}{)} \PY{o}{\PYZlt{}} \PY{l+m+mi}{30}
         \PY{k}{assert} \PY{n}{longest\PYZus{}run\PYZus{}simulations}\PY{o}{.}\PY{n}{max}\PY{p}{(}\PY{p}{)} \PY{o}{\PYZgt{}}\PY{o}{=} \PY{l+m+mi}{0}
\end{Verbatim}


    Use \texttt{bootstrap\_longest\_run} and the longest run statistic to
answer the following question: Is Klay's performance against the Indiana
Pacers indicative of hot hands? Support your answer with:

\begin{enumerate}
\def\labelenumi{\arabic{enumi}.}
\tightlist
\item
  A plot of the observed statistic against its (bootstrapped) sampling
  distribution. In this plot, each possible value of longest streak
  length should get its own bin, centered at its value. Restrict the
  x-axis to the interval {[}0, 20{]}.
\item
  A p-value compared to significance level 0.05
\item
  A sentence describing how the p-value should be interpreted.
\end{enumerate}

    \begin{Verbatim}[commandchars=\\\{\}]
{\color{incolor}In [{\color{incolor}29}]:} \PY{c+c1}{\PYZsh{} YOUR CODE HERE}
         \PY{n}{plt1}\PY{o}{=}\PY{n}{sns}\PY{o}{.}\PY{n}{distplot}\PY{p}{(}\PY{n}{longest\PYZus{}run\PYZus{}simulations}\PY{p}{,} \PY{n}{kde}\PY{o}{=}\PY{k+kc}{True}\PY{p}{)}
         \PY{n}{plt1}\PY{o}{.}\PY{n}{axes}\PY{o}{.}\PY{n}{set\PYZus{}xlim}\PY{p}{(}\PY{l+m+mi}{0}\PY{p}{,} \PY{l+m+mi}{20}\PY{p}{)}
         \PY{n}{plt1}\PY{o}{.}\PY{n}{axvline}\PY{p}{(}\PY{n}{find\PYZus{}longest\PYZus{}run}\PY{p}{(}\PY{n}{klay\PYZus{}example}\PY{p}{)}\PY{p}{[}\PY{l+m+mi}{0}\PY{p}{]}\PY{p}{,} \PY{l+m+mi}{0}\PY{p}{,}\PY{l+m+mi}{1}\PY{p}{,} \PY{n}{color} \PY{o}{=} \PY{l+s+s2}{\PYZdq{}}\PY{l+s+s2}{red}\PY{l+s+s2}{\PYZdq{}}\PY{p}{)}
         \PY{n+nb}{print}\PY{p}{(}\PY{l+s+s2}{\PYZdq{}}\PY{l+s+s2}{p\PYZhy{}value = }\PY{l+s+s2}{\PYZdq{}}\PY{o}{+}\PY{n+nb}{str}\PY{p}{(}\PY{n+nb}{sum}\PY{p}{(}\PY{n}{longest\PYZus{}run\PYZus{}simulations}\PY{o}{.}\PY{n}{tolist}\PY{p}{(}\PY{p}{)} \PY{o}{\PYZgt{}}\PY{o}{=} \PY{n}{find\PYZus{}longest\PYZus{}run}\PY{p}{(}\PY{n}{klay\PYZus{}example}\PY{p}{)}\PY{p}{[}\PY{l+m+mi}{0}\PY{p}{]}\PY{p}{)}\PY{o}{/}\PY{l+m+mi}{100}\PY{p}{)}\PY{p}{)}
         \PY{n+nb}{print}\PY{p}{(}\PY{l+s+s2}{\PYZdq{}}\PY{l+s+s2}{p\PYZhy{}value \PYZgt{}\PYZgt{} 0.05}\PY{l+s+s2}{\PYZdq{}}\PY{p}{)}
         \PY{c+c1}{\PYZsh{}raise NotImplementedError()}
\end{Verbatim}


    \begin{Verbatim}[commandchars=\\\{\}]
p-value = 0.9
p-value >> 0.05

    \end{Verbatim}

    \begin{center}
    \adjustimage{max size={0.9\linewidth}{0.9\paperheight}}{output_45_1.png}
    \end{center}
    { \hspace*{\fill} \\}
    
    Klay's preformance is against the existance of "hot hands". This is
because the chance that Klay's game had 6 or more consecutive scores is
89\% from the bootstrap above. This means it is not a rare event that
Klay would have scored 6 or more consecutive scores, since the p-value
\textgreater{}\textgreater{} 5\%.

    \subsection{A Different Statistic}\label{a-different-statistic}

Arguably, the longest run isn't a particularly good test-statistic for
capturing what people mean when they say "hot hands".

Let's try a test-statistics that captures the essence of "hot-hands" a
bit more. We're now going to explore a well-known approach proposed by
\href{https://en.wikipedia.org/wiki/Amos_Tversky}{Amos Tversky} and his
collaborators. The hot hand of Tversky is similar to the notion of being
"on fire" in the old arcade game
\href{https://www.youtube.com/watch?v=ipzstdPtxNw}{NBA Jam}. In that
game, if you make 3 shots in a row with a player, your player would be
on fire (with flame sprites!). While on fire (until a miss), the player
has an inflated probability of making shots.

The statistic to capture this affect, called \(T_{k, make}\), is easy to
compute:

\begin{align*}
  T_{k, make}
  & = \hat{\mathbb{P}}(\text{Make next shot}\ \vert\ \text{Made last $k$ shots})\\
  & = \frac{\#\{\text{Streaks of $k+1$ makes in a row}\}}{\#\{\text{Streaks of $k$ makes in a row preceeding an attempt}\}}
\end{align*}

If \(T_{k, make}\) is especially high, then we might say that our player
is experiencing a hot hand.

A similar statistic can try to capture a cold hand reversal:

\begin{align*}
  T_{k, miss}
  & = \hat{\mathbb{P}}(\text{Make next shot}\ \vert\ \text{Missed last $k$ shots})\\
  & = \frac{\#\{\text{Streaks of $k$ misses followed by make}\}}{\#\{\text{Streaks of $k$ misses in a row preceeding an attempt}\}}
\end{align*}

Note: If the value of \(T_{k, miss}\) is especially high, this doesn't
mean the player is expected to miss a bunch of shots in a row, instead
we'd say that they tend to see reversals in their streaks.

\paragraph{Problem 10 {[}10pts{]}}\label{problem-10-10pts}

Start by writing a utility function \texttt{count\_conditionally}, which
takes a \texttt{pd.Series} of shot sequence strings, a
\textbf{conditioning set}, and an \textbf{event}, and returns a series
of the count of the the number of times that the event follows the
conditioning set in each shot sequence string.

\textbf{Example Behavior 1:}

If we call
\texttt{count\_conditionally({[}\textquotesingle{}111111\textquotesingle{},\ \textquotesingle{}01111100111\textquotesingle{}{]},\ \textquotesingle{}111\textquotesingle{},\ \textquotesingle{}0\textquotesingle{})},
we are counting the number of times that the event \texttt{0} follows
\texttt{111} in each string. In this case, the function would return
\texttt{pd.Series({[}0,\ 1{]})}.

\textbf{Example Behavior 2:}

If we call
\texttt{count\_conditionally({[}\textquotesingle{}111111\textquotesingle{},\ \textquotesingle{}01111100111\textquotesingle{}{]},\ \textquotesingle{}111\textquotesingle{},\ \textquotesingle{}1\textquotesingle{})},
we are counting the number of times that the event \texttt{1} follows
\texttt{111} in each string. In this case, the function would return
\texttt{pd.Series({[}3,\ 2{]})}. Note that events can overlap, e.g.
\texttt{111111} has 3 occurrences of the event \texttt{1} that follow
the condition \texttt{111}: {111}{1}11, 1{111}{1}1, 11{111}{1}.

As with \texttt{count\_runs} and \texttt{find\_longest\_run}, for
convenience, your \texttt{count\_conditionally} function should handle a
string input corresponding to a single shot sequence as well.

\textbf{Hint}: You should be able to recycle ideas from
\texttt{count\_runs}.

    \begin{Verbatim}[commandchars=\\\{\}]
{\color{incolor}In [{\color{incolor}30}]:} \PY{k}{def} \PY{n+nf}{count\PYZus{}conditionally}\PY{p}{(}\PY{n}{shot\PYZus{}sequences}\PY{p}{,} \PY{n}{conditioning\PYZus{}set}\PY{p}{,} \PY{n}{event}\PY{o}{=}\PY{l+s+s1}{\PYZsq{}}\PY{l+s+s1}{1}\PY{l+s+s1}{\PYZsq{}}\PY{p}{)}\PY{p}{:}
             \PY{l+s+sd}{\PYZdq{}\PYZdq{}\PYZdq{}}
         \PY{l+s+sd}{    shot\PYZus{}sequences: pd.Series (string) of shot strings for a set of games or a single string}
         \PY{l+s+sd}{       to be coerced into a pd.Series}
         \PY{l+s+sd}{    conditioning\PYZus{}set: string or regex pattern representing the conditioning set}
         \PY{l+s+sd}{    event: string or regex pattern representing the event of interest}
         \PY{l+s+sd}{    }
         \PY{l+s+sd}{    return: pd.Series of the number of times event occured after the}
         \PY{l+s+sd}{       conditioning set in each game}
         \PY{l+s+sd}{    \PYZdq{}\PYZdq{}\PYZdq{}}
             
             \PY{c+c1}{\PYZsh{} YOUR CODE HERE}
             \PY{k}{return} \PY{n}{pd}\PY{o}{.}\PY{n}{Series}\PY{p}{(}\PY{p}{[}\PY{n+nb}{len}\PY{p}{(}\PY{n}{re}\PY{o}{.}\PY{n}{findall}\PY{p}{(}\PY{l+s+s1}{\PYZsq{}}\PY{l+s+s1}{(?\PYZlt{}=}\PY{l+s+s1}{\PYZsq{}}\PY{o}{+}\PY{n}{conditioning\PYZus{}set}\PY{o}{+}\PY{l+s+s1}{\PYZsq{}}\PY{l+s+s1}{)}\PY{l+s+s1}{\PYZsq{}}\PY{o}{+}\PY{n}{event}\PY{p}{,} \PY{n}{i}\PY{p}{)}\PY{p}{)} \PY{k}{for} \PY{n}{i} \PY{o+ow}{in} \PY{n}{pd}\PY{o}{.}\PY{n}{Series}\PY{p}{(}\PY{n}{shot\PYZus{}sequences}\PY{p}{)}\PY{p}{]}\PY{p}{)}
             \PY{c+c1}{\PYZsh{}raise NotImplementedError()}
\end{Verbatim}


    \begin{Verbatim}[commandchars=\\\{\}]
{\color{incolor}In [{\color{incolor}31}]:} \PY{k}{assert} \PY{n+nb}{isinstance}\PY{p}{(}\PY{n}{count\PYZus{}conditionally}\PY{p}{(}\PY{n}{pd}\PY{o}{.}\PY{n}{Series}\PY{p}{(}\PY{n}{klay\PYZus{}example}\PY{p}{)}\PY{p}{,} \PY{l+s+s1}{\PYZsq{}}\PY{l+s+s1}{11}\PY{l+s+s1}{\PYZsq{}}\PY{p}{)}\PY{p}{,} \PY{n}{pd}\PY{o}{.}\PY{n}{Series}\PY{p}{)}\PY{p}{,} \PYZbs{}
         \PY{l+s+s1}{\PYZsq{}}\PY{l+s+s1}{count\PYZus{}conditionally should return a pd.Series}\PY{l+s+s1}{\PYZsq{}}
\end{Verbatim}


    \begin{Verbatim}[commandchars=\\\{\}]
{\color{incolor}In [{\color{incolor}32}]:} \PY{c+c1}{\PYZsh{} Bah, test it yourself}
\end{Verbatim}


    \begin{Verbatim}[commandchars=\\\{\}]
{\color{incolor}In [{\color{incolor}33}]:} \PY{c+c1}{\PYZsh{} Nobody\PYZsq{}s home}
\end{Verbatim}


    \subsubsection{Worked examples}\label{worked-examples}

Read this section carefully. It will probably take some time to digest,
but it's a very valuable lesson in statistics that we'd like you to
absorb.

We'll look at the \(T_{k, make}\) statistic to make sure we understand
what it is, as well as what we might expect under the null vs. hot hands
hypothesis.

\paragraph{Example 1}\label{example-1}

Let's first consider a worked out example of computing \(T_{3, make}\),
the observed rate of success following a streak of 3 makes. We'll use
\texttt{111110001110} in our example. Looking at the string carefully,
we see that the condition \texttt{111} occurs 4 times. Of the 4
occurrences, 2 are followed by a make, and 2 are followed by a miss.
Thus \(T_{3, make}\) for \texttt{111110001110} is 0.5. Another way of
putting this is that
\texttt{count\_conditionally(\textquotesingle{}111110001110\textquotesingle{},\ \textquotesingle{}111\textquotesingle{},\ \textquotesingle{}1\textquotesingle{})}
returns the value 2 out of a possible maximum value of 4, and thus
\(T_{3, make}\) is 0.5.

\paragraph{Example 2}\label{example-2}

As another example, let's consider \(T_{3, make}\) for
\texttt{111110001110111}. In this case, the condition \texttt{111}
occurs 5 times. However, we will not count the last \texttt{111} as a
condition set, because there is no opportunity to flip again. We call
this last occurrence of \texttt{111} an \textbf{unrealized conditioning
set}. Of the remaining 4 occurrences, 2 are followed by a make, and 2
are followed by a miss. Thus \(T_{3, make}\) for
\texttt{111110001110111} is also 0.5. Another way of putting this is
that
\texttt{count\_conditionally(\textquotesingle{}111110001110111\textquotesingle{},\ \textquotesingle{}111\textquotesingle{},\ \textquotesingle{}1\textquotesingle{})}
returns the value 2 out of a possible maximum value of 4, and thus
\(T_{3, make}\) is 0.5.

\paragraph{Check your understanding}\label{check-your-understanding}

Compute \(T_{4, make}\) for \texttt{00000111100001111111000111} assuming
the probability the player makes a shot is 75\%. Click here to show the
answer Note that in \texttt{00000111100001111111000111}, there are 5
occurrences of \texttt{1111}, and the last one is not at the end, so is
not an unrealized conditioning set and thus all 5 conditioning sets are
under consideration. Of those 5 occurrences, the first is followed by a
miss, the second by a make, the third by a make, the fourth is followed
by a make, and the fifth is followed by a miss. Thus for this sequence,
\(T_{4, make}\) is 3/5 = 0.6.

Now that you know how to compute \(T_{k, make}\), let's reiterate that
it tells us the observed probability that we will make the next shot,
given that we have made the previous k shots. That is for the sequence
\texttt{00000111100001111111000111}, the fact that \(T_{4, make}\) is
equal to 0.6 means that the \textbf{observed probability} of making a
shot after 4 shots in a row is 60\%.

\subsubsection{\texorpdfstring{Computing the Expectated Value of
\(T_{1, make}\)}{Computing the Expectated Value of T\_\{1, make\}}}\label{computing-the-expectated-value-of-t_1-make}

Consider \(T_{1, make}\), i.e. the observed probability that you make a
shot, given that your last shot was also a make. Before continuing, make
sure you can compute that \(T_{1, make}\) of \texttt{1110} is
\(\frac{2}{3}\).

We ultimately want to take player shot sequences and compute
\(T_{k, make}\), so it'd be a good idea if we know what to expect under
the null hypothesis.

\paragraph{Thought Exercise}\label{thought-exercise}

Suppose that a given player's probability of making a shot is 50\%, and
that they make exactly 4 shots. Under the null hypothesis (hot hands
does not exist), give your guess for the expected value of
\(T_{1, make}\). Supply your answer by setting the variable
\texttt{ev\_tk1\_make}.

In other words, if you pick \texttt{ev\_tk1\_make\ =\ 0.8}, you're
saying that for a shot sequence of four shots for a player with 50\%
accuracy, under the null hypothesis (hot hands doesn't exist) you expect
that you will observe the player making 80\% of their shots that follow
a make.

    \begin{Verbatim}[commandchars=\\\{\}]
{\color{incolor}In [{\color{incolor}34}]:} \PY{c+c1}{\PYZsh{} Doesn\PYZsq{}t matter what you write.  This is just to keep you honest about}
         \PY{c+c1}{\PYZsh{} your intuition}
         \PY{n}{ev\PYZus{}tk1\PYZus{}make} \PY{o}{=} \PY{l+m+mf}{0.5}
         
         \PY{c+c1}{\PYZsh{} YOUR CODE HERE}
         \PY{c+c1}{\PYZsh{}raise NotImplementedError()}
\end{Verbatim}


    \begin{Verbatim}[commandchars=\\\{\}]
{\color{incolor}In [{\color{incolor}35}]:} \PY{k}{assert} \PY{l+m+mi}{0} \PY{o}{\PYZlt{}}\PY{o}{=} \PY{n}{ev\PYZus{}tk1\PYZus{}make} \PY{o}{\PYZlt{}}\PY{o}{=} \PY{l+m+mi}{1}
\end{Verbatim}


    We're guessing that you picked \texttt{ev\_tk1\_make\ =\ 0.5}, which is
a great guess! It seems clear that if shots are made independently, the
chance of making a basket is 50\%. While the OVERALL probability is
50\%, the CONDITIONAL probability will not be 50\%. In other words the
expected value of \(T_{1, make}\) will not be 0.5 under the null
hypothesis if we're considering a shot sequence of 4 shots with 50\%
probability.

How can this be? We will show it to be true by enumerating all the
possibilities. Run the cell below to list the four different
possibilities for our shot sequences, with the value of \(T_{1, make}\)
for each sequence in the rightmost column. \texttt{n11} is how many
times our conditioning set is realized and followed by a 1, and
\texttt{n10} is how many times our conditioning set is realized and
followed by a 0.

    \begin{Verbatim}[commandchars=\\\{\}]
{\color{incolor}In [{\color{incolor}36}]:} \PY{k}{def} \PY{n+nf}{iterable\PYZus{}to\PYZus{}string}\PY{p}{(}\PY{n}{iterable}\PY{p}{)}\PY{p}{:}
             \PY{k}{return} \PY{l+s+s1}{\PYZsq{}}\PY{l+s+s1}{\PYZsq{}}\PY{o}{.}\PY{n}{join}\PY{p}{(}\PY{n+nb}{map}\PY{p}{(}\PY{n+nb}{str}\PY{p}{,} \PY{n}{iterable}\PY{p}{)}\PY{p}{)}
         
         \PY{n}{example} \PY{o}{=} \PY{n}{pd}\PY{o}{.}\PY{n}{DataFrame}\PY{p}{(}\PY{p}{\PYZob{}}
          \PY{l+s+s1}{\PYZsq{}}\PY{l+s+s1}{sequence}\PY{l+s+s1}{\PYZsq{}}\PY{p}{:} \PY{p}{[}\PY{n}{iterable\PYZus{}to\PYZus{}string}\PY{p}{(}\PY{n}{s}\PY{p}{)} \PY{k}{for} \PY{n}{s} \PY{o+ow}{in} \PY{n}{itertools}\PY{o}{.}\PY{n}{product}\PY{p}{(}\PY{l+s+s1}{\PYZsq{}}\PY{l+s+s1}{10}\PY{l+s+s1}{\PYZsq{}}\PY{p}{,} \PY{n}{repeat}\PY{o}{=}\PY{l+m+mi}{4}\PY{p}{)}\PY{p}{]}   
         \PY{p}{\PYZcb{}}\PY{p}{)}
         \PY{n}{example}\PY{p}{[}\PY{l+s+s1}{\PYZsq{}}\PY{l+s+s1}{n11}\PY{l+s+s1}{\PYZsq{}}\PY{p}{]} \PY{o}{=} \PY{n}{count\PYZus{}conditionally}\PY{p}{(}\PY{n}{example}\PY{p}{[}\PY{l+s+s1}{\PYZsq{}}\PY{l+s+s1}{sequence}\PY{l+s+s1}{\PYZsq{}}\PY{p}{]}\PY{p}{,} \PY{l+s+s1}{\PYZsq{}}\PY{l+s+s1}{1}\PY{l+s+s1}{\PYZsq{}}\PY{p}{,} \PY{l+s+s1}{\PYZsq{}}\PY{l+s+s1}{1}\PY{l+s+s1}{\PYZsq{}}\PY{p}{)}
         \PY{n}{example}\PY{p}{[}\PY{l+s+s1}{\PYZsq{}}\PY{l+s+s1}{n10}\PY{l+s+s1}{\PYZsq{}}\PY{p}{]} \PY{o}{=} \PY{n}{count\PYZus{}conditionally}\PY{p}{(}\PY{n}{example}\PY{p}{[}\PY{l+s+s1}{\PYZsq{}}\PY{l+s+s1}{sequence}\PY{l+s+s1}{\PYZsq{}}\PY{p}{]}\PY{p}{,} \PY{l+s+s1}{\PYZsq{}}\PY{l+s+s1}{1}\PY{l+s+s1}{\PYZsq{}}\PY{p}{,} \PY{l+s+s1}{\PYZsq{}}\PY{l+s+s1}{0}\PY{l+s+s1}{\PYZsq{}}\PY{p}{)}
         \PY{n}{example}\PY{p}{[}\PY{l+s+s1}{\PYZsq{}}\PY{l+s+s1}{tk1}\PY{l+s+s1}{\PYZsq{}}\PY{p}{]} \PY{o}{=} \PY{p}{(}\PY{n}{example}\PY{p}{[}\PY{l+s+s1}{\PYZsq{}}\PY{l+s+s1}{n11}\PY{l+s+s1}{\PYZsq{}}\PY{p}{]} \PY{o}{/} \PY{p}{(}\PY{n}{example}\PY{p}{[}\PY{l+s+s1}{\PYZsq{}}\PY{l+s+s1}{n11}\PY{l+s+s1}{\PYZsq{}}\PY{p}{]} \PY{o}{+} \PY{n}{example}\PY{p}{[}\PY{l+s+s1}{\PYZsq{}}\PY{l+s+s1}{n10}\PY{l+s+s1}{\PYZsq{}}\PY{p}{]}\PY{p}{)}\PY{p}{)}\PY{o}{.}\PY{n}{round}\PY{p}{(}\PY{l+m+mi}{2}\PY{p}{)}
         
         \PY{n}{example}
\end{Verbatim}


\begin{Verbatim}[commandchars=\\\{\}]
{\color{outcolor}Out[{\color{outcolor}36}]:}    sequence  n11  n10   tk1
         0      1111    3    0  1.00
         1      1110    2    1  0.67
         2      1101    1    1  0.50
         3      1100    1    1  0.50
         4      1011    1    1  0.50
         5      1010    0    2  0.00
         6      1001    0    1  0.00
         7      1000    0    1  0.00
         8      0111    2    0  1.00
         9      0110    1    1  0.50
         10     0101    0    1  0.00
         11     0100    0    1  0.00
         12     0011    1    0  1.00
         13     0010    0    1  0.00
         14     0001    0    0   NaN
         15     0000    0    0   NaN
\end{Verbatim}
            
    Since each sequence is equally likely (you should prove this to
yourself!), each of the possible observations for \(T_{1, make}\) have
the same probability, and we can just take the arithmetic average of
\texttt{tk1}, dropping any undefined proportions, to get the expected
value.

    \begin{Verbatim}[commandchars=\\\{\}]
{\color{incolor}In [{\color{incolor}37}]:} \PY{n}{ev\PYZus{}tk1\PYZus{}actual} \PY{o}{=} \PY{n}{example}\PY{p}{[}\PY{l+s+s1}{\PYZsq{}}\PY{l+s+s1}{tk1}\PY{l+s+s1}{\PYZsq{}}\PY{p}{]}\PY{o}{.}\PY{n}{dropna}\PY{p}{(}\PY{p}{)}\PY{o}{.}\PY{n}{mean}\PY{p}{(}\PY{p}{)}\PY{o}{.}\PY{n}{round}\PY{p}{(}\PY{l+m+mi}{2}\PY{p}{)}
         \PY{n+nb}{print}\PY{p}{(}\PY{n}{f}\PY{l+s+s1}{\PYZsq{}}\PY{l+s+s1}{The expected value of the conditional proportion is }\PY{l+s+si}{\PYZob{}ev\PYZus{}tk1\PYZus{}actual\PYZcb{}}\PY{l+s+s1}{\PYZsq{}}\PY{p}{)}
\end{Verbatim}


    \begin{Verbatim}[commandchars=\\\{\}]
The expected value of the conditional proportion is 0.4

    \end{Verbatim}

    Surprised? We certainly were! You can do a similar analysis of
\(T_{k, miss}\) to find that it is greater than 0.5, meaning the
expected proportion of streak reversals (a 1 after a sequence of
consecutive 0s) is higher than 0.5, the overall probability of getting a
1!

Differently put, if we label the sequence as \(s_1, s_2, s_3, s_4\), we
can write the proportion as:

\[\begin{align}
T_{1, make}
&= \hat{\mathbb{P}} (\text{Get a 1 given that previous result was 1}) \\
&= \hat{\mathbb{P}}(s_i = 1\ \vert\ s_{i-1}=1) \\
&= \dfrac{n_{11}}{n_{10}+n_{11}}
\end{align}\]

It may seem like \(\mathbb{E}\left\lbrack T_{1, make} \right\rbrack\)
should really be 0.5 if the chance of making a shot is 50\%, but the
table above shows that this is NOT the case. The observed chance of
getting a make given that you just got a make is actually 40\% when you
have a sequence of 4 shots.

Notice that the table above is also enough to fully describe the
sampling distribution of \(T_{1, make}\) for a player with an accuracy
of 50\%. Below is a plot of the probability distribution. There are only
4 possible values for \(T_{1, make}\):
\(0, \frac{1}{2}, \frac{2}{3}, \text{and } 1\).

    \begin{Verbatim}[commandchars=\\\{\}]
{\color{incolor}In [{\color{incolor}38}]:} \PY{n}{sns}\PY{o}{.}\PY{n}{distplot}\PY{p}{(}\PY{n}{example}\PY{p}{[}\PY{l+s+s1}{\PYZsq{}}\PY{l+s+s1}{tk1}\PY{l+s+s1}{\PYZsq{}}\PY{p}{]}\PY{o}{.}\PY{n}{dropna}\PY{p}{(}\PY{p}{)}\PY{p}{,}
                      \PY{n}{kde}\PY{o}{=}\PY{k+kc}{False}\PY{p}{,} \PY{n}{bins}\PY{o}{=}\PY{n}{np}\PY{o}{.}\PY{n}{arange}\PY{p}{(}\PY{o}{\PYZhy{}}\PY{l+m+mf}{0.005}\PY{p}{,} \PY{l+m+mf}{1.015}\PY{p}{,} \PY{o}{.}\PY{l+m+mi}{01}\PY{p}{)}\PY{p}{,} \PY{n}{norm\PYZus{}hist}\PY{o}{=}\PY{k+kc}{True}\PY{p}{)}
         \PY{n}{plt}\PY{o}{.}\PY{n}{xticks}\PY{p}{(}\PY{n}{np}\PY{o}{.}\PY{n}{arange}\PY{p}{(}\PY{l+m+mi}{0}\PY{p}{,} \PY{l+m+mf}{1.1}\PY{p}{,} \PY{l+m+mf}{0.1}\PY{p}{)}\PY{p}{)}
         \PY{n}{plt}\PY{o}{.}\PY{n}{title}\PY{p}{(}\PY{l+s+s1}{\PYZsq{}}\PY{l+s+s1}{Sampling Distribution of \PYZdl{}T\PYZus{}}\PY{l+s+s1}{\PYZob{}}\PY{l+s+s1}{1, make\PYZcb{}\PYZdl{}}\PY{l+s+se}{\PYZbs{}n}\PY{l+s+s1}{ k=1, 4 shots, 50}\PY{l+s+s1}{\PYZpc{}}\PY{l+s+s1}{ Accuracy}\PY{l+s+s1}{\PYZsq{}}\PY{p}{)}
         \PY{n}{plt}\PY{o}{.}\PY{n}{xlabel}\PY{p}{(}\PY{l+s+s1}{\PYZsq{}}\PY{l+s+s1}{\PYZdl{}T\PYZus{}}\PY{l+s+s1}{\PYZob{}}\PY{l+s+s1}{1,make\PYZcb{}\PYZdl{}}\PY{l+s+s1}{\PYZsq{}}\PY{p}{)}
         \PY{n}{plt}\PY{o}{.}\PY{n}{ylabel}\PY{p}{(}\PY{l+s+s1}{\PYZsq{}}\PY{l+s+s1}{Probability (Approx }\PY{l+s+s1}{\PYZpc{}}\PY{l+s+s1}{)}\PY{l+s+s1}{\PYZsq{}}\PY{p}{)}\PY{p}{;}
\end{Verbatim}


    \begin{center}
    \adjustimage{max size={0.9\linewidth}{0.9\paperheight}}{output_60_0.png}
    \end{center}
    { \hspace*{\fill} \\}
    
    \paragraph{Problem 11 {[}5pts{]}}\label{problem-11-5pts}

Recall that in the example above, we were conditioning on runs of length
1 and a player that shoots 4 times with an accuracy of 50\%. Calculate
the expected proportion of makes conditioned on runs of length 2 (i.e.
\(T_{2, make}\)) when the player shoots 16 times with an accuracy of
50\%.

    \begin{Verbatim}[commandchars=\\\{\}]
{\color{incolor}In [{\color{incolor}39}]:} \PY{k}{def} \PY{n+nf}{findt2make}\PY{p}{(}\PY{n}{seq}\PY{p}{)}\PY{p}{:}
             \PY{n}{x} \PY{o}{=} \PY{n}{count\PYZus{}conditionally}\PY{p}{(}\PY{n}{seq}\PY{p}{,} \PY{l+s+s1}{\PYZsq{}}\PY{l+s+s1}{11}\PY{l+s+s1}{\PYZsq{}}\PY{p}{,} \PY{l+s+s1}{\PYZsq{}}\PY{l+s+s1}{1}\PY{l+s+s1}{\PYZsq{}}\PY{p}{)}
             \PY{n}{y} \PY{o}{=} \PY{n}{count\PYZus{}conditionally}\PY{p}{(}\PY{n}{seq}\PY{p}{,} \PY{l+s+s1}{\PYZsq{}}\PY{l+s+s1}{11}\PY{l+s+s1}{\PYZsq{}}\PY{p}{,} \PY{l+s+s1}{\PYZsq{}}\PY{l+s+s1}{0}\PY{l+s+s1}{\PYZsq{}}\PY{p}{)}
             \PY{k}{return} \PY{p}{(}\PY{n}{x} \PY{o}{/} \PY{p}{(}\PY{n}{x} \PY{o}{+} \PY{n}{y}\PY{p}{)}\PY{p}{)}
         \PY{n}{finallst} \PY{o}{=} \PY{p}{[}\PY{p}{]}
         \PY{n}{possible} \PY{o}{=} \PY{l+m+mi}{2}\PY{o}{*}\PY{o}{*}\PY{l+m+mi}{16}
         \PY{k}{for} \PY{n}{i} \PY{o+ow}{in} \PY{n}{np}\PY{o}{.}\PY{n}{arange}\PY{p}{(}\PY{n}{possible}\PY{p}{)}\PY{p}{:}
             \PY{n}{x}\PY{o}{=}\PY{n}{findt2make}\PY{p}{(}\PY{n}{np}\PY{o}{.}\PY{n}{binary\PYZus{}repr}\PY{p}{(}\PY{n}{i}\PY{p}{,} \PY{n}{width}\PY{o}{=}\PY{l+m+mi}{16}\PY{p}{)}\PY{p}{)}
             \PY{k}{if} \PY{o+ow}{not} \PY{n}{pd}\PY{o}{.}\PY{n}{isna}\PY{p}{(}\PY{n}{x}\PY{p}{)}\PY{p}{[}\PY{l+m+mi}{0}\PY{p}{]}\PY{p}{:}
                 \PY{n}{finallst} \PY{o}{+}\PY{o}{=} \PY{p}{[}\PY{n}{x}\PY{p}{[}\PY{l+m+mi}{0}\PY{p}{]}\PY{p}{]}
         \PY{n}{expected\PYZus{}proportion} \PY{o}{=} \PY{n}{np}\PY{o}{.}\PY{n}{mean}\PY{p}{(}\PY{n}{finallst}\PY{p}{)}
         \PY{n}{expected\PYZus{}proportion}
         \PY{c+c1}{\PYZsh{} YOUR CODE HERE}
         \PY{c+c1}{\PYZsh{}raise NotImplementedError()}
\end{Verbatim}


\begin{Verbatim}[commandchars=\\\{\}]
{\color{outcolor}Out[{\color{outcolor}39}]:} 0.39962552841631144
\end{Verbatim}
            
    \begin{Verbatim}[commandchars=\\\{\}]
{\color{incolor}In [{\color{incolor}40}]:} \PY{k}{assert} \PY{l+m+mi}{0} \PY{o}{\PYZlt{}}\PY{o}{=} \PY{n}{expected\PYZus{}proportion} \PY{o}{\PYZlt{}}\PY{o}{=} \PY{l+m+mi}{1}\PY{p}{,} \PYZbs{}
         \PY{l+s+s1}{\PYZsq{}}\PY{l+s+s1}{The expected proportion should be between 0 and 1.}\PY{l+s+s1}{\PYZsq{}}
\end{Verbatim}


    \paragraph{Problem 12 {[}5pts{]}}\label{problem-12-5pts}

Plot the sampling distribution of \(T_{2, make}\) of a player who shoots
16 times with an accuracy of 50\%. You should be able to reuse your work
from the last problem.

    \begin{Verbatim}[commandchars=\\\{\}]
{\color{incolor}In [{\color{incolor}41}]:} \PY{c+c1}{\PYZsh{} YOUR CODE HERE}
         \PY{n}{sns}\PY{o}{.}\PY{n}{distplot}\PY{p}{(}\PY{n}{finallst}\PY{p}{,} \PY{n}{kde}\PY{o}{=}\PY{k+kc}{False}\PY{p}{,} \PY{n}{bins}\PY{o}{=}\PY{n}{np}\PY{o}{.}\PY{n}{arange}\PY{p}{(}\PY{o}{\PYZhy{}}\PY{l+m+mf}{0.005}\PY{p}{,} \PY{l+m+mf}{1.015}\PY{p}{,} \PY{o}{.}\PY{l+m+mi}{01}\PY{p}{)}\PY{p}{,} \PY{n}{norm\PYZus{}hist}\PY{o}{=}\PY{k+kc}{True}\PY{p}{)}
         \PY{n}{plt}\PY{o}{.}\PY{n}{xticks}\PY{p}{(}\PY{n}{np}\PY{o}{.}\PY{n}{arange}\PY{p}{(}\PY{l+m+mi}{0}\PY{p}{,} \PY{l+m+mf}{1.1}\PY{p}{,} \PY{l+m+mf}{0.1}\PY{p}{)}\PY{p}{)}
         \PY{n}{plt}\PY{o}{.}\PY{n}{title}\PY{p}{(}\PY{l+s+s1}{\PYZsq{}}\PY{l+s+s1}{Sampling Distribution of \PYZdl{}T\PYZus{}}\PY{l+s+s1}{\PYZob{}}\PY{l+s+s1}{2, make\PYZcb{}\PYZdl{}}\PY{l+s+se}{\PYZbs{}n}\PY{l+s+s1}{ k=1, 16 shots, 50}\PY{l+s+s1}{\PYZpc{}}\PY{l+s+s1}{ Accuracy}\PY{l+s+s1}{\PYZsq{}}\PY{p}{)}
         \PY{n}{plt}\PY{o}{.}\PY{n}{xlabel}\PY{p}{(}\PY{l+s+s1}{\PYZsq{}}\PY{l+s+s1}{\PYZdl{}T\PYZus{}}\PY{l+s+s1}{\PYZob{}}\PY{l+s+s1}{2,make\PYZcb{}\PYZdl{}}\PY{l+s+s1}{\PYZsq{}}\PY{p}{)}
         \PY{n}{plt}\PY{o}{.}\PY{n}{ylabel}\PY{p}{(}\PY{l+s+s1}{\PYZsq{}}\PY{l+s+s1}{Probability (Approx }\PY{l+s+s1}{\PYZpc{}}\PY{l+s+s1}{)}\PY{l+s+s1}{\PYZsq{}}\PY{p}{)}\PY{p}{;}
         \PY{c+c1}{\PYZsh{}raise NotImplementedError()}
\end{Verbatim}


    \begin{center}
    \adjustimage{max size={0.9\linewidth}{0.9\paperheight}}{output_65_0.png}
    \end{center}
    { \hspace*{\fill} \\}
    
    It turns out that the logic for calculating the exact sampling
distribution for \(T_{k, make}\) when a player has an accuracy other
than 50\% is a little more complicated than we're immediately equipped
to deal with in this class. This might seem like our analysis is going
to be doomed. If we can't compute \(T_{k, make}\) to expect under the
null hypothesis for a given player, how will be able to recognize a
\(T_{k, make}\) value that indicates that the hot hands hypothesis is
true?

Luckily, we don't have to, because we have a tool that will allow us to
approximate the sampling distribution under the null hypothesis: the
bootstrap. The key observation is that the bootstrap procedure naturally
preserves the player's overall shot accuracy in a given game. We'll use
the bootstrap in a little while at the very end of this homework.

In short, all that hard work we just did to compute the exact sampling
distribution of \(T_{k, make}\) isn't going to play a role in our
analysis. Nonetheless, we felt it was important to really dig in and
gain intuition on this test statistic.

\subsection{The "Tversky Statistic" for Hot Hand
Detection}\label{the-tversky-statistic-for-hot-hand-detection}

It turns out that simply measuring \(T_k = T_{k, make}\) isn't as useful
as the "Tversky statistic" for hot-hand detection, defined as

\[T_k = T_{k, make} - T_{k, miss}\]

The original inspiration for this statistic was to measure
hot-handedness by comparing the proportion of times a player continued a
success streak against their propensity to reverse a string of misses.
As we saw above, computing the expected value of \(T_{k, make}\) is hard
and the results are counterintuitive. We're not going to formally
explore the expected value of \(T_k\), but you are free and encouraged
to do so.

We will, however, mention that for reasons similar to our analysis in
the previous sections, despite most people's initial intuition that the
expected value of \(T_k\) should be zero, this statistic has its
sampling distribution centered around a value less than 0.

\paragraph{Problem 13 {[}5pts{]}}\label{problem-13-5pts}

The Tversky statistic is sometimes undefined (has no valid value). In
our analysis, we will be discarding sequences where \(T_k\) is
undefined. The reason is that it doesn't make sense to count cases where
the conditioning set isn't present. Specifically describe the two cases
where \(T_k\) is undefined.

    One case is if the player only shot less than or equal to k times.
Another case is if the player misses all of their shots.

    \paragraph{Problem 14 {[}5pts{]}}\label{problem-14-5pts}

Write a function \texttt{calc\_tk\_stat} that can take a
\texttt{pd.Series} of shot strings and return their Tversky statistics.
If the statistic is undefined, return \texttt{NaN}.

    \begin{Verbatim}[commandchars=\\\{\}]
{\color{incolor}In [{\color{incolor}42}]:} \PY{k}{def} \PY{n+nf}{calc\PYZus{}tk\PYZus{}stat}\PY{p}{(}\PY{n}{games}\PY{p}{,} \PY{n}{k}\PY{p}{)}\PY{p}{:}
             \PY{l+s+sd}{\PYZdq{}\PYZdq{}\PYZdq{}}
         \PY{l+s+sd}{    Computes the tversky statistic for hot hands}
         \PY{l+s+sd}{        }
         \PY{l+s+sd}{    games: pd.Series (string) shot data for a set of games}
         \PY{l+s+sd}{    k: int, conditioning set length; number of misses/hits to condition on}
         \PY{l+s+sd}{    \PYZdq{}\PYZdq{}\PYZdq{}}
             \PY{n}{n11} \PY{o}{=} \PY{n}{count\PYZus{}conditionally}\PY{p}{(}\PY{n}{games}\PY{p}{,} \PY{l+s+s2}{\PYZdq{}}\PY{l+s+s2}{1}\PY{l+s+s2}{\PYZdq{}}\PY{o}{*}\PY{n}{k}\PY{p}{,} \PY{l+s+s1}{\PYZsq{}}\PY{l+s+s1}{1}\PY{l+s+s1}{\PYZsq{}}\PY{p}{)}
             \PY{n}{n10} \PY{o}{=} \PY{n}{count\PYZus{}conditionally}\PY{p}{(}\PY{n}{games}\PY{p}{,} \PY{l+s+s2}{\PYZdq{}}\PY{l+s+s2}{1}\PY{l+s+s2}{\PYZdq{}}\PY{o}{*}\PY{n}{k}\PY{p}{,} \PY{l+s+s1}{\PYZsq{}}\PY{l+s+s1}{0}\PY{l+s+s1}{\PYZsq{}}\PY{p}{)}
             \PY{n}{n01} \PY{o}{=} \PY{n}{count\PYZus{}conditionally}\PY{p}{(}\PY{n}{games}\PY{p}{,} \PY{l+s+s2}{\PYZdq{}}\PY{l+s+s2}{0}\PY{l+s+s2}{\PYZdq{}}\PY{o}{*}\PY{n}{k}\PY{p}{,} \PY{l+s+s1}{\PYZsq{}}\PY{l+s+s1}{1}\PY{l+s+s1}{\PYZsq{}}\PY{p}{)}
             \PY{n}{n00} \PY{o}{=} \PY{n}{count\PYZus{}conditionally}\PY{p}{(}\PY{n}{games}\PY{p}{,} \PY{l+s+s2}{\PYZdq{}}\PY{l+s+s2}{0}\PY{l+s+s2}{\PYZdq{}}\PY{o}{*}\PY{n}{k}\PY{p}{,} \PY{l+s+s1}{\PYZsq{}}\PY{l+s+s1}{0}\PY{l+s+s1}{\PYZsq{}}\PY{p}{)}
             \PY{k}{return} \PY{n}{n11}\PY{o}{/}\PY{p}{(}\PY{n}{n11}\PY{o}{+}\PY{n}{n10}\PY{p}{)} \PY{o}{\PYZhy{}} \PY{n}{n01}\PY{o}{/}\PY{p}{(}\PY{n}{n00}\PY{o}{+}\PY{n}{n01}\PY{p}{)}
         \PY{c+c1}{\PYZsh{} YOUR CODE HERE}
         \PY{n}{calc\PYZus{}tk\PYZus{}stat}\PY{p}{(}\PY{l+s+s1}{\PYZsq{}}\PY{l+s+s1}{1110100110000011}\PY{l+s+s1}{\PYZsq{}}\PY{p}{,} \PY{l+m+mi}{2}\PY{p}{)}
         \PY{c+c1}{\PYZsh{}raise NotImplementedError()}
\end{Verbatim}


\begin{Verbatim}[commandchars=\\\{\}]
{\color{outcolor}Out[{\color{outcolor}42}]:} 0   -0.066667
         dtype: float64
\end{Verbatim}
            
    \begin{Verbatim}[commandchars=\\\{\}]
{\color{incolor}In [{\color{incolor}43}]:} \PY{k}{assert} \PY{n}{np}\PY{o}{.}\PY{n}{isclose}\PY{p}{(}\PY{n}{calc\PYZus{}tk\PYZus{}stat}\PY{p}{(}\PY{n}{pd}\PY{o}{.}\PY{n}{Series}\PY{p}{(}\PY{p}{[}\PY{l+s+s1}{\PYZsq{}}\PY{l+s+s1}{1110100110000011}\PY{l+s+s1}{\PYZsq{}}\PY{p}{]}\PY{p}{)}\PY{p}{,} \PY{l+m+mi}{2}\PY{p}{)}\PY{p}{,} \PY{n}{pd}\PY{o}{.}\PY{n}{Series}\PY{p}{(}\PY{p}{[}\PY{o}{\PYZhy{}}\PY{l+m+mi}{1}\PY{o}{/}\PY{l+m+mi}{15}\PY{p}{]}\PY{p}{)}\PY{p}{)}\PY{p}{,} \PYZbs{}
         \PY{l+s+s1}{\PYZsq{}}\PY{l+s+s1}{T\PYZus{}2 for 1110100110000011 is \PYZhy{}1/15}\PY{l+s+s1}{\PYZsq{}}
\end{Verbatim}


    \subsubsection{Statistically Testing the Null
Hypothesis}\label{statistically-testing-the-null-hypothesis}

Now we return to the question of whether or not Thompson has hot hands.
Under the hypothesis that he does have hot hands, Klay Thompson has a
higher chance of making shots when he has recently made shots. Under the
null hypothesis, his chance of making a shot is independent of recent
successes.

Run the cell below, which we'll use to load all of Klay Thompson's data.

Assuming you've correctly read in \texttt{shot\_data},
\texttt{klay\_data} is a \texttt{pd.Series} containing Klay Thompson's
shot records for the 2016-2017 season for all games (not just the game
where he got 60 points).

    \begin{Verbatim}[commandchars=\\\{\}]
{\color{incolor}In [{\color{incolor}44}]:} \PY{n}{klay\PYZus{}data} \PY{o}{=} \PY{n}{shot\PYZus{}data}\PY{o}{.}\PY{n}{loc}\PY{p}{[}\PY{n}{pd}\PY{o}{.}\PY{n}{IndexSlice}\PY{p}{[}\PY{p}{:}\PY{p}{,} \PY{l+s+s1}{\PYZsq{}}\PY{l+s+s1}{K. Thompson}\PY{l+s+s1}{\PYZsq{}}\PY{p}{]}\PY{p}{,} \PY{l+s+s1}{\PYZsq{}}\PY{l+s+s1}{shots}\PY{l+s+s1}{\PYZsq{}}\PY{p}{]}
         \PY{n}{klay\PYZus{}data}\PY{o}{.}\PY{n}{head}\PY{p}{(}\PY{l+m+mi}{5}\PY{p}{)}
\end{Verbatim}


\begin{Verbatim}[commandchars=\\\{\}]
{\color{outcolor}Out[{\color{outcolor}44}]:} game          player     
         201701200HOU  K. Thompson      1000111100010000100
         201703180GSW  K. Thompson        01100110001110011
         201611040LAL  K. Thompson       000100000010011000
         201611100DEN  K. Thompson         0011011011100011
         201612300GSW  K. Thompson    101010110111111110000
         Name: shots, dtype: object
\end{Verbatim}
            
    \begin{Verbatim}[commandchars=\\\{\}]
{\color{incolor}In [{\color{incolor}45}]:} \PY{k}{assert} \PY{n+nb}{isinstance}\PY{p}{(}\PY{n}{klay\PYZus{}data}\PY{p}{,} \PY{n}{pd}\PY{o}{.}\PY{n}{Series}\PY{p}{)}\PY{p}{,} \PYZbs{}
         \PY{l+s+s1}{\PYZsq{}}\PY{l+s+s1}{klay\PYZus{}data should be a pd.Series}\PY{l+s+s1}{\PYZsq{}}
         \PY{k}{assert} \PY{n}{klay\PYZus{}data}\PY{o}{.}\PY{n}{shape}\PY{p}{[}\PY{l+m+mi}{0}\PY{p}{]} \PY{o}{==} \PY{l+m+mi}{95}\PY{p}{,} \PYZbs{}
         \PY{l+s+s1}{\PYZsq{}}\PY{l+s+s1}{You have too few observations (should be 95)}\PY{l+s+s1}{\PYZsq{}}
         \PY{k}{assert} \PY{l+s+s1}{\PYZsq{}}\PY{l+s+s1}{0000010110101}\PY{l+s+s1}{\PYZsq{}} \PY{o+ow}{in} \PY{n}{klay\PYZus{}data}\PY{o}{.}\PY{n}{values}\PY{p}{,} \PYZbs{}
         \PY{l+s+s1}{\PYZsq{}}\PY{l+s+s1}{000001011010 is missing from your data}\PY{l+s+s1}{\PYZsq{}}
         \PY{k}{assert} \PY{n}{klay\PYZus{}data}\PY{o}{.}\PY{n}{apply}\PY{p}{(}\PY{k}{lambda} \PY{n}{x}\PY{p}{:} \PY{n+nb}{sum}\PY{p}{(}\PY{p}{[}\PY{n+nb}{int}\PY{p}{(}\PY{n}{n}\PY{p}{)} \PY{k}{for} \PY{n}{n} \PY{o+ow}{in} \PY{n}{x}\PY{p}{]}\PY{p}{)}\PY{p}{)}\PY{o}{.}\PY{n}{sum}\PY{p}{(}\PY{p}{)} \PY{o}{==} \PY{l+m+mi}{950}\PY{p}{,} \PYZbs{}
         \PY{l+s+s1}{\PYZsq{}}\PY{l+s+s1}{You failed the checksum}\PY{l+s+s1}{\PYZsq{}}
\end{Verbatim}


    \paragraph{Problem 15 {[}10pts{]}}\label{problem-15-10pts}

To help carry out the analysis at scale, write a function
\texttt{calc\_p\_values} that can take a \texttt{pd.Series} of test
statistics (one for each game) and compare it to a \texttt{pd.DataFrame}
of simulated statistics. In the \texttt{pd.DataFrame}, each row
corresponds to the a game, so the shape will be (number of games, number
of bootstrap replications). You may assume \texttt{observed\_statistics}
does not contain any \texttt{NaNs}; however,
\texttt{simulated\_statistics} may have some.

\textbf{Example Behavior}

If our observed statistics are
\texttt{pd.Series({[}0.5,\ 0.35,\ 0.4{]})} and our simulated statistics
are a Dataframe with the values:

\begin{verbatim}
0.1, 0.3, 0.4, 0.6, 0.4, 0.6, 0.8, 0.9
0.3, NaN, 0.7, 0.1, 0.3, 0.1, 0.8, 0.6
0.3, 0.7, 0.1, 0.6, 0.7, NaN, NaN, 0.2
\end{verbatim}

Then your function should return
\texttt{pd.Series({[}4/8,\ 3/7,\ 3/6{]})}, e.g. the number of simulated
statistics that matched or exceeded the observed statistic were 4 out of
a possible 8.

    \begin{Verbatim}[commandchars=\\\{\}]
{\color{incolor}In [{\color{incolor}46}]:} \PY{k}{def} \PY{n+nf}{calc\PYZus{}p\PYZus{}values}\PY{p}{(}\PY{n}{observed\PYZus{}statistics}\PY{p}{,} \PY{n}{simulated\PYZus{}statistics}\PY{p}{)}\PY{p}{:}
             \PY{l+s+sd}{\PYZdq{}\PYZdq{}\PYZdq{}}
         \PY{l+s+sd}{    observed\PYZus{}statistics: pd.Series (float), test statistics for each game }
         \PY{l+s+sd}{    simulated\PYZus{}statistics: pd.DataFrame, rows represent games, columns contain }
         \PY{l+s+sd}{        test statistics simulated under the null hypothesis }
         \PY{l+s+sd}{    }
         \PY{l+s+sd}{    return: pd.Series (float), p\PYZhy{}values for every game between 0 and 1}
         \PY{l+s+sd}{    \PYZdq{}\PYZdq{}\PYZdq{}}
             \PY{n}{res}\PY{o}{=}\PY{p}{[}\PY{p}{]}
             \PY{k}{for} \PY{n}{i} \PY{o+ow}{in} \PY{n}{np}\PY{o}{.}\PY{n}{arange}\PY{p}{(}\PY{n+nb}{len}\PY{p}{(}\PY{n}{observed\PYZus{}statistics}\PY{p}{)}\PY{p}{)}\PY{p}{:}
                 \PY{n}{temp} \PY{o}{=} \PY{p}{[}\PY{p}{]}
                 \PY{k}{for} \PY{n}{j} \PY{o+ow}{in} \PY{n}{simulated\PYZus{}statistics}\PY{o}{.}\PY{n}{loc}\PY{p}{[}\PY{n}{i}\PY{p}{]}\PY{p}{:}
                     \PY{k}{if} \PY{o+ow}{not} \PY{n}{pd}\PY{o}{.}\PY{n}{isna}\PY{p}{(}\PY{n}{j}\PY{p}{)} \PY{o+ow}{and} \PY{o+ow}{not} \PY{n}{pd}\PY{o}{.}\PY{n}{isna}\PY{p}{(}\PY{n}{observed\PYZus{}statistics}\PY{p}{[}\PY{n}{i}\PY{p}{]}\PY{p}{)}\PY{p}{:}
                         \PY{n}{temp} \PY{o}{+}\PY{o}{=} \PY{p}{[}\PY{n}{j}\PY{o}{\PYZgt{}}\PY{o}{=}\PY{n}{observed\PYZus{}statistics}\PY{p}{[}\PY{n}{i}\PY{p}{]}\PY{p}{]}
                 \PY{k}{if} \PY{n}{pd}\PY{o}{.}\PY{n}{isna}\PY{p}{(}\PY{n}{observed\PYZus{}statistics}\PY{p}{[}\PY{n}{i}\PY{p}{]}\PY{p}{)}\PY{p}{:}
                     \PY{n}{res} \PY{o}{+}\PY{o}{=} \PY{p}{[}\PY{n}{np}\PY{o}{.}\PY{n}{nan}\PY{p}{]}
                 \PY{k}{else}\PY{p}{:}
                     \PY{n}{res} \PY{o}{+}\PY{o}{=} \PY{p}{[}\PY{n}{np}\PY{o}{.}\PY{n}{mean}\PY{p}{(}\PY{n}{temp}\PY{p}{)}\PY{p}{]}
             \PY{k}{return} \PY{n}{pd}\PY{o}{.}\PY{n}{Series}\PY{p}{(}\PY{n}{res}\PY{p}{)}
             \PY{c+c1}{\PYZsh{} YOUR CODE HERE}
             \PY{c+c1}{\PYZsh{}raise NotImplementedError()}
\end{Verbatim}


    \begin{Verbatim}[commandchars=\\\{\}]
{\color{incolor}In [{\color{incolor}47}]:} \PY{n}{pv\PYZus{}obstat} \PY{o}{=} \PY{n}{pd}\PY{o}{.}\PY{n}{Series}\PY{p}{(}\PY{p}{[}\PY{l+m+mf}{0.5}\PY{p}{,} \PY{l+m+mf}{0.35}\PY{p}{,} \PY{l+m+mf}{0.4}\PY{p}{]}\PY{p}{)}
         \PY{n}{pv\PYZus{}simstat} \PY{o}{=} \PY{n}{pd}\PY{o}{.}\PY{n}{DataFrame}\PY{p}{(}\PY{n}{columns}\PY{o}{=}\PY{n+nb}{list}\PY{p}{(}\PY{n+nb}{range}\PY{p}{(}\PY{l+m+mi}{100}\PY{p}{)}\PY{p}{)}\PY{p}{,} \PY{n}{index}\PY{o}{=}\PY{n+nb}{list}\PY{p}{(}\PY{n+nb}{range}\PY{p}{(}\PY{l+m+mi}{3}\PY{p}{)}\PY{p}{)}\PY{p}{)}
         \PY{n}{pv\PYZus{}simstat}\PY{o}{.}\PY{n}{loc}\PY{p}{[}\PY{l+m+mi}{0}\PY{p}{]} \PY{o}{=} \PY{n}{pd}\PY{o}{.}\PY{n}{Series}\PY{p}{(}\PY{p}{[}\PY{l+m+mf}{0.1}\PY{p}{,} \PY{l+m+mf}{0.3}\PY{p}{,} \PY{l+m+mf}{0.4}\PY{p}{,} \PY{l+m+mf}{0.6}\PY{p}{,} \PY{l+m+mf}{0.4}\PY{p}{,} \PY{l+m+mf}{0.6}\PY{p}{,} \PY{l+m+mf}{0.8}\PY{p}{,} \PY{l+m+mf}{0.9}\PY{p}{]}\PY{p}{)}
         \PY{n}{pv\PYZus{}simstat}\PY{o}{.}\PY{n}{loc}\PY{p}{[}\PY{l+m+mi}{1}\PY{p}{]} \PY{o}{=} \PY{n}{pd}\PY{o}{.}\PY{n}{Series}\PY{p}{(}\PY{p}{[}\PY{l+m+mf}{0.3}\PY{p}{,} \PY{n}{np}\PY{o}{.}\PY{n}{nan}\PY{p}{,} \PY{l+m+mf}{0.7}\PY{p}{,} \PY{l+m+mf}{0.1}\PY{p}{,} \PY{l+m+mf}{0.3}\PY{p}{,} \PY{l+m+mf}{0.1}\PY{p}{,} \PY{l+m+mf}{0.8}\PY{p}{,} \PY{l+m+mf}{0.6}\PY{p}{]}\PY{p}{)}
         \PY{n}{pv\PYZus{}simstat}\PY{o}{.}\PY{n}{loc}\PY{p}{[}\PY{l+m+mi}{2}\PY{p}{]} \PY{o}{=} \PY{n}{pd}\PY{o}{.}\PY{n}{Series}\PY{p}{(}\PY{p}{[}\PY{l+m+mf}{0.3}\PY{p}{,} \PY{l+m+mf}{0.7}\PY{p}{,} \PY{l+m+mf}{0.1}\PY{p}{,} \PY{l+m+mf}{0.6}\PY{p}{,} \PY{l+m+mf}{0.7}\PY{p}{,} \PY{n}{np}\PY{o}{.}\PY{n}{nan}\PY{p}{,} \PY{n}{np}\PY{o}{.}\PY{n}{nan}\PY{p}{,} \PY{l+m+mf}{0.2}\PY{p}{]}\PY{p}{)}
         
         \PY{k}{assert} \PY{n+nb}{isinstance}\PY{p}{(}\PY{n}{calc\PYZus{}p\PYZus{}values}\PY{p}{(}\PY{n}{pv\PYZus{}obstat}\PY{p}{,} \PY{n}{pv\PYZus{}simstat}\PY{p}{)}\PY{p}{,} \PY{n}{pd}\PY{o}{.}\PY{n}{Series}\PY{p}{)}
         \PY{k}{assert} \PY{n}{calc\PYZus{}p\PYZus{}values}\PY{p}{(}\PY{n}{pv\PYZus{}obstat}\PY{p}{,} \PY{n}{pv\PYZus{}simstat}\PY{p}{)}\PY{o}{.}\PY{n}{equals}\PY{p}{(}\PY{n}{pd}\PY{o}{.}\PY{n}{Series}\PY{p}{(}\PY{p}{[}\PY{l+m+mi}{4}\PY{o}{/}\PY{l+m+mi}{8}\PY{p}{,} \PY{l+m+mi}{3}\PY{o}{/}\PY{l+m+mi}{7}\PY{p}{,} \PY{l+m+mi}{3}\PY{o}{/}\PY{l+m+mi}{6}\PY{p}{]}\PY{p}{)}\PY{p}{)}
\end{Verbatim}


    \begin{Verbatim}[commandchars=\\\{\}]
{\color{incolor}In [{\color{incolor}48}]:} \PY{c+c1}{\PYZsh{} No admittance except on party business}
\end{Verbatim}


    \paragraph{Problem 16 {[}Graded in the Synthesis
Portion{]}}\label{problem-16-graded-in-the-synthesis-portion}

Carry out bootstrap hypothesis tests for all 95 records in
\texttt{klay\_data} for conditioning sets of length \(k=1,2,3\). Use
10000 bootstrap replicates to approximate the sampling distribution in
each test. You will report your results in the following section.
Technically, we should be worried about
\href{https://en.wikipedia.org/wiki/Multiple_comparisons_problem}{multiple
testing issues}, but you can ignore them in your analysis.

For the cell below, there is no specific structure to the output that
you must produce. However, your code should compute at least: * The
observed Tversky statistic for each of the 95 games. For example, for
\(k = 1\), for game
\texttt{\textquotesingle{}201610250GSW\textquotesingle{}}, the observed
Tversky statistic is exactly \texttt{-0.250000}. * The number of
observations that had to be discarded due to an undefined Tversky
statistic. For example, the game
\texttt{\textquotesingle{}201610250GSW\textquotesingle{}} with shot
sequence \texttt{\textquotesingle{}0000010110101\textquotesingle{}} has
an undefined Tversky statistic for \(k = 3\). * The p-values for each of
the 95 games. For eaxmple, for \(k=1\), for game
\texttt{\textquotesingle{}201610250GSW\textquotesingle{}}, the p-value
should be approximately 0.75. * The number of games whose p-values were
significant at the 5\% level. For example, you might find that for
\(k=1\), 90 out of 95 games have a p-value of less than 0.05, which
would be strong evidence of the hot hands effect.

You'll compile the results of your findings in the next and final
section of this homework.

    \begin{Verbatim}[commandchars=\\\{\}]
{\color{incolor}In [{\color{incolor}49}]:} \PY{n}{klay\PYZus{}data}\PY{o}{.}\PY{n}{loc}\PY{p}{[}\PY{l+s+s1}{\PYZsq{}}\PY{l+s+s1}{201612050GSW}\PY{l+s+s1}{\PYZsq{}}\PY{p}{]}
\end{Verbatim}


\begin{Verbatim}[commandchars=\\\{\}]
{\color{outcolor}Out[{\color{outcolor}49}]:} player
         K. Thompson    11011110010111111001110111101110111101010101
         Name: shots, dtype: object
\end{Verbatim}
            
    \begin{Verbatim}[commandchars=\\\{\}]
{\color{incolor}In [{\color{incolor}50}]:} \PY{c+c1}{\PYZsh{} YOUR CODE HERE}
         \PY{k}{def} \PY{n+nf}{bootstrapingData}\PY{p}{(}\PY{n}{game}\PY{p}{,} \PY{n}{num\PYZus{}iter}\PY{o}{=}\PY{l+m+mi}{10000}\PY{p}{)}\PY{p}{:}
             \PY{l+s+sd}{\PYZdq{}\PYZdq{}\PYZdq{}}
         \PY{l+s+sd}{    Returns num\PYZus{}iter games that have been bootstrapped from the provided game}
         \PY{l+s+sd}{    \PYZdq{}\PYZdq{}\PYZdq{}}
             \PY{n}{attempts} \PY{o}{=} \PY{n+nb}{len}\PY{p}{(}\PY{n}{game}\PY{p}{)}
             \PY{n}{makes} \PY{o}{=} \PY{n+nb}{sum}\PY{p}{(}\PY{p}{[}\PY{n+nb}{int}\PY{p}{(}\PY{n}{i}\PY{p}{)} \PY{k}{for} \PY{n}{i} \PY{o+ow}{in} \PY{n}{game}\PY{p}{]}\PY{p}{)}
             \PY{n}{accuracy} \PY{o}{=} \PY{n}{makes}\PY{o}{/}\PY{n}{attempts}
             \PY{n}{res}\PY{o}{=}\PY{p}{[}\PY{p}{]}
             \PY{k}{for} \PY{n}{\PYZus{}} \PY{o+ow}{in} \PY{n}{np}\PY{o}{.}\PY{n}{arange}\PY{p}{(}\PY{n}{num\PYZus{}iter}\PY{p}{)}\PY{p}{:}
                 \PY{n}{inpt} \PY{o}{=} \PY{l+s+s2}{\PYZdq{}}\PY{l+s+s2}{\PYZdq{}}
                 \PY{k}{for} \PY{n}{\PYZus{}\PYZus{}} \PY{o+ow}{in} \PY{n}{np}\PY{o}{.}\PY{n}{arange}\PY{p}{(}\PY{n+nb}{len}\PY{p}{(}\PY{n}{game}\PY{p}{)}\PY{p}{)}\PY{p}{:}
                     \PY{k}{if} \PY{n}{np}\PY{o}{.}\PY{n}{random}\PY{o}{.}\PY{n}{uniform}\PY{p}{(}\PY{p}{)} \PY{o}{\PYZlt{}} \PY{n}{accuracy}\PY{p}{:}
                         \PY{n}{inpt} \PY{o}{+}\PY{o}{=} \PY{l+s+s2}{\PYZdq{}}\PY{l+s+s2}{1}\PY{l+s+s2}{\PYZdq{}}
                     \PY{k}{else}\PY{p}{:}
                         \PY{n}{inpt} \PY{o}{+}\PY{o}{=} \PY{l+s+s2}{\PYZdq{}}\PY{l+s+s2}{0}\PY{l+s+s2}{\PYZdq{}}
                 \PY{n}{res} \PY{o}{+}\PY{o}{=} \PY{p}{[}\PY{n}{inpt}\PY{p}{]}
             \PY{k}{return} \PY{n}{res}
         
         \PY{n}{simstat} \PY{o}{=} \PY{p}{[}\PY{p}{]}
         \PY{n}{simstat} \PY{o}{+}\PY{o}{=} \PY{p}{[}\PY{n}{pd}\PY{o}{.}\PY{n}{DataFrame}\PY{p}{(}\PY{n}{columns}\PY{o}{=}\PY{n+nb}{list}\PY{p}{(}\PY{n+nb}{range}\PY{p}{(}\PY{l+m+mi}{10000}\PY{p}{)}\PY{p}{)}\PY{p}{)}\PY{p}{]}
         \PY{n}{simstat} \PY{o}{+}\PY{o}{=} \PY{p}{[}\PY{n}{pd}\PY{o}{.}\PY{n}{DataFrame}\PY{p}{(}\PY{n}{columns}\PY{o}{=}\PY{n+nb}{list}\PY{p}{(}\PY{n+nb}{range}\PY{p}{(}\PY{l+m+mi}{10000}\PY{p}{)}\PY{p}{)}\PY{p}{)}\PY{p}{]}
         \PY{n}{simstat} \PY{o}{+}\PY{o}{=} \PY{p}{[}\PY{n}{pd}\PY{o}{.}\PY{n}{DataFrame}\PY{p}{(}\PY{n}{columns}\PY{o}{=}\PY{n+nb}{list}\PY{p}{(}\PY{n+nb}{range}\PY{p}{(}\PY{l+m+mi}{10000}\PY{p}{)}\PY{p}{)}\PY{p}{)}\PY{p}{]}
         
         \PY{n}{observed} \PY{o}{=} \PY{p}{[}\PY{p}{]}
         \PY{n}{observed} \PY{o}{+}\PY{o}{=} \PY{p}{[}\PY{n}{pd}\PY{o}{.}\PY{n}{Series}\PY{p}{(}\PY{p}{)}\PY{p}{]}
         \PY{n}{observed} \PY{o}{+}\PY{o}{=} \PY{p}{[}\PY{n}{pd}\PY{o}{.}\PY{n}{Series}\PY{p}{(}\PY{p}{)}\PY{p}{]}
         \PY{n}{observed} \PY{o}{+}\PY{o}{=} \PY{p}{[}\PY{n}{pd}\PY{o}{.}\PY{n}{Series}\PY{p}{(}\PY{p}{)}\PY{p}{]}
         
         \PY{n}{pacersobsstat} \PY{o}{=} \PY{l+m+mi}{0}
         \PY{n}{pacersbootstat} \PY{o}{=} \PY{l+m+mi}{0}
         
         \PY{k}{for} \PY{n}{game} \PY{o+ow}{in} \PY{n}{klay\PYZus{}data}\PY{p}{:}
             \PY{n}{bootstrap\PYZus{}of\PYZus{}game} \PY{o}{=} \PY{n}{bootstrapingData}\PY{p}{(}\PY{n}{game}\PY{p}{)}
             \PY{k}{for} \PY{n}{k} \PY{o+ow}{in} \PY{p}{[}\PY{l+m+mi}{1}\PY{p}{,} \PY{l+m+mi}{2}\PY{p}{,} \PY{l+m+mi}{3}\PY{p}{]}\PY{p}{:}
                 \PY{n}{stat} \PY{o}{=} \PY{n}{calc\PYZus{}tk\PYZus{}stat}\PY{p}{(}\PY{n}{game}\PY{p}{,} \PY{n}{k}\PY{p}{)}
                 \PY{k}{if} \PY{o+ow}{not} \PY{n}{pd}\PY{o}{.}\PY{n}{isna}\PY{p}{(}\PY{n}{stat}\PY{p}{[}\PY{l+m+mi}{0}\PY{p}{]}\PY{p}{)}\PY{p}{:}
                     \PY{n}{calcstats} \PY{o}{=} \PY{n}{calc\PYZus{}tk\PYZus{}stat}\PY{p}{(}\PY{n}{bootstrap\PYZus{}of\PYZus{}game}\PY{p}{,} \PY{n}{k}\PY{p}{)}
                     \PY{k}{if} \PY{n}{k} \PY{o}{==} \PY{l+m+mi}{1} \PY{o+ow}{and} \PY{n}{game} \PY{o}{==} \PY{l+s+s1}{\PYZsq{}}\PY{l+s+s1}{11011110010111111001110111101110111101010101}\PY{l+s+s1}{\PYZsq{}}\PY{p}{:}
                         \PY{n}{pacersobsstat} \PY{o}{=} \PY{n}{stat}\PY{p}{[}\PY{l+m+mi}{0}\PY{p}{]}
                         \PY{n}{pacersbootstat} \PY{o}{=} \PY{n}{calcstats}
                     \PY{n}{observed}\PY{p}{[}\PY{n}{k}\PY{o}{\PYZhy{}}\PY{l+m+mi}{1}\PY{p}{]} \PY{o}{=} \PY{n}{observed}\PY{p}{[}\PY{n}{k}\PY{o}{\PYZhy{}}\PY{l+m+mi}{1}\PY{p}{]}\PY{o}{.}\PY{n}{append}\PY{p}{(}\PY{n}{stat}\PY{p}{)}
                     \PY{n}{simstat}\PY{p}{[}\PY{n}{k}\PY{o}{\PYZhy{}}\PY{l+m+mi}{1}\PY{p}{]} \PY{o}{=} \PY{n}{simstat}\PY{p}{[}\PY{n}{k}\PY{o}{\PYZhy{}}\PY{l+m+mi}{1}\PY{p}{]}\PY{o}{.}\PY{n}{append}\PY{p}{(}\PY{n}{calcstats}\PY{p}{,} \PY{n}{ignore\PYZus{}index} \PY{o}{=} \PY{k+kc}{True}\PY{p}{)}
         
         \PY{c+c1}{\PYZsh{}raise NotImplementedError()}
\end{Verbatim}


    \begin{Verbatim}[commandchars=\\\{\}]
{\color{incolor}In [{\color{incolor}51}]:} \PY{n}{p\PYZus{}vals} \PY{o}{=} \PY{p}{[}\PY{p}{]}
         \PY{n}{p\PYZus{}vals} \PY{o}{+}\PY{o}{=} \PY{p}{[}\PY{n}{calc\PYZus{}p\PYZus{}values}\PY{p}{(}\PY{n}{observed}\PY{p}{[}\PY{l+m+mi}{0}\PY{p}{]}\PY{o}{.}\PY{n}{reset\PYZus{}index}\PY{p}{(}\PY{n}{drop} \PY{o}{=} \PY{k+kc}{True}\PY{p}{)}\PY{p}{,} \PY{n}{simstat}\PY{p}{[}\PY{l+m+mi}{0}\PY{p}{]}\PY{p}{)}\PY{p}{]}
         \PY{n}{p\PYZus{}vals} \PY{o}{+}\PY{o}{=} \PY{p}{[}\PY{n}{calc\PYZus{}p\PYZus{}values}\PY{p}{(}\PY{n}{observed}\PY{p}{[}\PY{l+m+mi}{1}\PY{p}{]}\PY{o}{.}\PY{n}{reset\PYZus{}index}\PY{p}{(}\PY{n}{drop} \PY{o}{=} \PY{k+kc}{True}\PY{p}{)}\PY{p}{,} \PY{n}{simstat}\PY{p}{[}\PY{l+m+mi}{1}\PY{p}{]}\PY{p}{)}\PY{p}{]}
         \PY{n}{p\PYZus{}vals} \PY{o}{+}\PY{o}{=} \PY{p}{[}\PY{n}{calc\PYZus{}p\PYZus{}values}\PY{p}{(}\PY{n}{observed}\PY{p}{[}\PY{l+m+mi}{2}\PY{p}{]}\PY{o}{.}\PY{n}{reset\PYZus{}index}\PY{p}{(}\PY{n}{drop} \PY{o}{=} \PY{k+kc}{True}\PY{p}{)}\PY{p}{,} \PY{n}{simstat}\PY{p}{[}\PY{l+m+mi}{2}\PY{p}{]}\PY{p}{)}\PY{p}{]}
         \PY{n+nb}{print}\PY{p}{(}\PY{n}{np}\PY{o}{.}\PY{n}{sum}\PY{p}{(}\PY{n}{p\PYZus{}vals}\PY{p}{[}\PY{l+m+mi}{0}\PY{p}{]} \PY{o}{\PYZlt{}}\PY{o}{=} \PY{l+m+mf}{0.05}\PY{p}{)}\PY{p}{)}
         \PY{n+nb}{print}\PY{p}{(}\PY{n}{np}\PY{o}{.}\PY{n}{sum}\PY{p}{(}\PY{n}{p\PYZus{}vals}\PY{p}{[}\PY{l+m+mi}{1}\PY{p}{]} \PY{o}{\PYZlt{}}\PY{o}{=} \PY{l+m+mf}{0.05}\PY{p}{)}\PY{p}{)}
         \PY{n+nb}{print}\PY{p}{(}\PY{n}{np}\PY{o}{.}\PY{n}{sum}\PY{p}{(}\PY{n}{p\PYZus{}vals}\PY{p}{[}\PY{l+m+mi}{2}\PY{p}{]} \PY{o}{\PYZlt{}}\PY{o}{=} \PY{l+m+mf}{0.05}\PY{p}{)}\PY{p}{)}
\end{Verbatim}


    \begin{Verbatim}[commandchars=\\\{\}]
5
4
2

    \end{Verbatim}

    \begin{Verbatim}[commandchars=\\\{\}]
{\color{incolor}In [{\color{incolor}52}]:} \PY{n+nb}{print}\PY{p}{(}\PY{n}{pacersobsstat}\PY{p}{)}
         \PY{n+nb}{print}\PY{p}{(}\PY{n}{pacersbootstat}\PY{o}{.}\PY{n}{dropna}\PY{p}{(}\PY{p}{)}\PY{p}{)}
\end{Verbatim}


    \begin{Verbatim}[commandchars=\\\{\}]
-0.212820512821
0       0.044444
1      -0.343750
2       0.119048
3       0.016340
4       0.117949
5       0.127451
6       0.022727
7      -0.228571
8      -0.070136
9      -0.011312
10     -0.024631
11      0.004630
12      0.022727
13     -0.008065
14      0.003968
15      0.118182
16     -0.040323
17      0.306452
18     -0.102564
19      0.357143
20     -0.264706
21      0.078571
22     -0.252841
23     -0.023810
24      0.187879
25     -0.433333
26     -0.099432
27      0.042857
28      0.042857
29      0.022727
          {\ldots}   
9970   -0.040323
9971   -0.059524
9972    0.085973
9973    0.085973
9974   -0.099432
9975    0.127451
9976    0.046798
9977    0.258621
9978   -0.155914
9979   -0.330952
9980   -0.023810
9981   -0.102564
9982    0.186275
9983   -0.303030
9984   -0.008523
9985    0.223118
9986   -0.200000
9987   -0.072581
9988    0.087879
9989   -0.042424
9990   -0.194444
9991   -0.094907
9992    0.081281
9993    0.145238
9994   -0.093596
9995    0.140000
9996    0.119048
9997   -0.040323
9998   -0.179487
9999    0.104167
Length: 10000, dtype: float64

    \end{Verbatim}

    \subsection{Synthesis}\label{synthesis}

Running the numerical computations in hypothesis testing is only part of
the battle. Convincing others of the validity of the analysis is just as
if not more important. Compile everything you have done/learned into a
miniature report. Describe how you used the Tversky statistic to test
whether or not Klay Thompson has hot hands. Your answer should follow
the structure given below. While we can provide you with an idea of
items you should definitely include in such a report, you will need to
supply the wording to concisely and convincingly tell the story.

\textbf{Note:} DO NOT copy this cell using command mode. This will cause
the autograder to fail on your notebook. You may, however, double click
on the cell and copy its text.

\textbf{Data Generation Model}

We modeled Klay Thompson's shot record for each game as sequences of
\texttt{INSERT\ description\ of\ random\ variable} with the following
assumptions

\begin{itemize}
\tightlist
\item
  \texttt{INSERT\ Assumption\ 1}
\item
  ...
\end{itemize}

We realize that this ignores the following real-life issues

\begin{itemize}
\tightlist
\item
  \texttt{INSERT\ Issue\ 1}
\item
  ...
\end{itemize}

However, this analysis can be used as a baseline that we can compare
more complicated models to.

\textbf{Null Hypothesis}

Our null hypothesis is
\texttt{INSERT\ null\ hypothesis\ in\ plain\ English}. In terms of our
model, this means that
\texttt{INSERT\ mathematical\ implication\ of\ null\ hypothesis}.

\textbf{Test Statistic}

To test our hypothesis, we used the Tversky statistic, which can be
interpreted as \texttt{INSERT\ plain\ English\ description\ in\ words}.
This can be written mathematically as:

\[\text{INSERT LaTeX statistic = function of data}\]

\textbf{Results}

Looking Klay's December 5th game against the Pacers, we calculated a
p-value of \texttt{INSERT\ p-value} for \(k=1\), which
\texttt{CHOOSE\ ONE:\ is\ \ or\ is\ not} significant at the 5\% level.
This can be verified visually in the following plot.

\[\text{Insert plot of sampling distribution and observed statistic}\]

We go on to analyze all of Thompson's games and find that
\texttt{CHOOSE\ ONE:\ few\ or\ many} of the observations are significant
at the 5\% level for conditioning sets of length \(k=1,2,3\). The table
below shows the number of observations that we discarded due to the
statistic being undefined and the number that are significant at each
conditioning length.

\begin{longtable}[]{@{}lllll@{}}
\toprule
\begin{minipage}[b]{0.05\columnwidth}\raggedright\strut
Player\strut
\end{minipage} & \begin{minipage}[b]{0.05\columnwidth}\raggedright\strut
Number of Games\strut
\end{minipage} & \begin{minipage}[b]{0.05\columnwidth}\raggedright\strut
k\strut
\end{minipage} & \begin{minipage}[b]{0.05\columnwidth}\raggedright\strut
Number of Games Discarded\strut
\end{minipage} & \begin{minipage}[b]{0.05\columnwidth}\raggedright\strut
Number of Games Significant\strut
\end{minipage}\tabularnewline
\midrule
\endhead
\begin{minipage}[t]{0.05\columnwidth}\raggedright\strut
Thompson\strut
\end{minipage} & \begin{minipage}[t]{0.05\columnwidth}\raggedright\strut
95\strut
\end{minipage} & \begin{minipage}[t]{0.05\columnwidth}\raggedright\strut
1\strut
\end{minipage} & \begin{minipage}[t]{0.05\columnwidth}\raggedright\strut
\texttt{INSERT\ \#\ Dropped\ for\ k=1}\strut
\end{minipage} & \begin{minipage}[t]{0.05\columnwidth}\raggedright\strut
\texttt{INSERT\ \#\ Significant\ for\ k=1}\strut
\end{minipage}\tabularnewline
\begin{minipage}[t]{0.05\columnwidth}\raggedright\strut
-\strut
\end{minipage} & \begin{minipage}[t]{0.05\columnwidth}\raggedright\strut
2\strut
\end{minipage} & \begin{minipage}[t]{0.05\columnwidth}\raggedright\strut
\texttt{INSERT\ \#\ Dropped\ for\ k=2}\strut
\end{minipage} & \begin{minipage}[t]{0.05\columnwidth}\raggedright\strut
\texttt{INSERT\ \#\ Significant\ for\ k=2}\strut
\end{minipage}\tabularnewline
\begin{minipage}[t]{0.05\columnwidth}\raggedright\strut
-\strut
\end{minipage} & \begin{minipage}[t]{0.05\columnwidth}\raggedright\strut
3\strut
\end{minipage} & \begin{minipage}[t]{0.05\columnwidth}\raggedright\strut
\texttt{INSERT\ \#\ Dropped\ for\ k=3}\strut
\end{minipage} & \begin{minipage}[t]{0.05\columnwidth}\raggedright\strut
\texttt{INSERT\ \#\ Significant\ for\ k=3}\strut
\end{minipage}\tabularnewline
\bottomrule
\end{longtable}

    \paragraph{Data Generation Model
{[}8pts{]}}\label{data-generation-model-8pts}

    We modeled Klay Thompson's shot record for each game as a concatenated
string of 0s (misses) and 1s (makes) with the following assumptions

\begin{itemize}
\tightlist
\item
  The shots are made with no regards with the time between each shot
\item
  Scoring a 3 pointer, 2 pointer, or free throw all result in a '1'
\end{itemize}

We realize that this ignores the following real-life issues

\begin{itemize}
\tightlist
\item
  Distance of each shot
\item
  Opposing team's defenders
\item
  Player's teammates help the player score by restricting opposing team
\item
  The point score for each shot
\end{itemize}

However, this analysis can be used as a baseline that we can compare
more complicated models to.

    \paragraph{Null Hypothesis {[}5pts{]}}\label{null-hypothesis-5pts}

    Our null hypothesis is that the notion of "hot hands" are not in affect,
meaning a player is not more likely to make a shot given the player has
made the last previous shot(s). In terms of our model, this means that
\(\mathbb{P}(\)player makes a shot\() = \mathbb{P}(\)player makes a shot
\textbar{} player has made the last \(k\) shots\()\).

    \paragraph{Test Statistic {[}2pts{]}}\label{test-statistic-2pts}

    To test our hypothesis, we used the Tversky statistic, which can be
interpreted as the probability that a player makes a shot given they had
made the last k shots minus the probability that a player makes a shot
given they had missed the last k shots. This can be written
mathematically as:

\[T_K = T_{k,make} - T_{k,miss}\]
\[T_K = \mathbb{P}(S_i=1 | S_{i-1}, ..., S_{i-k} = 1) - \mathbb{P}(S_i=1 | S_{i-1}, ..., S_{i-k} = 0)\]

    \paragraph{Results {[}20pts{]}}\label{results-20pts}

    Looking Klay's December 5th game against the Pacers, we calculated a
p-value of 0.86 for \(k=1\), which is not significant at the 5\% level.
This can be verified visually in the following plot.

    \begin{Verbatim}[commandchars=\\\{\}]
{\color{incolor}In [{\color{incolor}53}]:} \PY{c+c1}{\PYZsh{} Plotting Code}
         \PY{n}{plt1}\PY{o}{=}\PY{n}{sns}\PY{o}{.}\PY{n}{distplot}\PY{p}{(}\PY{n}{pacersbootstat}\PY{o}{.}\PY{n}{dropna}\PY{p}{(}\PY{p}{)}\PY{p}{,} \PY{n}{kde}\PY{o}{=}\PY{k+kc}{True}\PY{p}{)}
         \PY{n}{plt1}\PY{o}{.}\PY{n}{set\PYZus{}title}\PY{p}{(}\PY{l+s+s2}{\PYZdq{}}\PY{l+s+s2}{Histogram of Sampling Distribution}\PY{l+s+s2}{\PYZdq{}}\PY{p}{)}
         \PY{c+c1}{\PYZsh{}plt1.axes.set\PYZus{}xlim()}
         \PY{n}{plt1}\PY{o}{.}\PY{n}{axvline}\PY{p}{(}\PY{n}{pacersobsstat}\PY{p}{,} \PY{l+m+mi}{0}\PY{p}{,}\PY{l+m+mi}{1}\PY{p}{,} \PY{n}{color} \PY{o}{=} \PY{l+s+s2}{\PYZdq{}}\PY{l+s+s2}{red}\PY{l+s+s2}{\PYZdq{}}\PY{p}{,} \PY{n}{label} \PY{o}{=} \PY{l+s+s2}{\PYZdq{}}\PY{l+s+s2}{observed statistic}\PY{l+s+s2}{\PYZdq{}}\PY{p}{)}
         \PY{n}{plt1}\PY{o}{.}\PY{n}{legend}\PY{p}{(}\PY{p}{)}
         \PY{n+nb}{print}\PY{p}{(}\PY{l+s+s2}{\PYZdq{}}\PY{l+s+s2}{p\PYZhy{}value = }\PY{l+s+s2}{\PYZdq{}}\PY{o}{+}\PY{n+nb}{str}\PY{p}{(}\PY{n+nb}{sum}\PY{p}{(}\PY{n}{pacersbootstat}\PY{o}{.}\PY{n}{dropna}\PY{p}{(}\PY{p}{)}\PY{o}{.}\PY{n}{tolist}\PY{p}{(}\PY{p}{)} \PY{o}{\PYZgt{}}\PY{o}{=} \PY{n}{pacersobsstat}\PY{p}{)}\PY{o}{/}\PY{n+nb}{len}\PY{p}{(}\PY{n}{pacersbootstat}\PY{o}{.}\PY{n}{dropna}\PY{p}{(}\PY{p}{)}\PY{p}{)}\PY{p}{)}\PY{p}{)}
         \PY{n+nb}{print}\PY{p}{(}\PY{l+s+s2}{\PYZdq{}}\PY{l+s+s2}{p\PYZhy{}value \PYZgt{}\PYZgt{} 0.05}\PY{l+s+s2}{\PYZdq{}}\PY{p}{)}
         \PY{c+c1}{\PYZsh{} YOUR CODE HERE}
         \PY{c+c1}{\PYZsh{}raise NotImplementedError()}
\end{Verbatim}


    \begin{Verbatim}[commandchars=\\\{\}]
p-value = 0.8974
p-value >> 0.05

    \end{Verbatim}

    \begin{center}
    \adjustimage{max size={0.9\linewidth}{0.9\paperheight}}{output_92_1.png}
    \end{center}
    { \hspace*{\fill} \\}
    
    We go on to analyze all of Thompson's games and find that few of the
observations are significant at the 5\% level for conditioning sets of
length \(k=1,2,3\). The table below shows the number of observations
that we discarded due to the statistic being undefined and the number
that are significant at each conditioning length.

\begin{longtable}[]{@{}lllll@{}}
\toprule
\begin{minipage}[b]{0.05\columnwidth}\raggedright\strut
Player\strut
\end{minipage} & \begin{minipage}[b]{0.05\columnwidth}\raggedright\strut
Number of Games\strut
\end{minipage} & \begin{minipage}[b]{0.05\columnwidth}\raggedright\strut
k\strut
\end{minipage} & \begin{minipage}[b]{0.05\columnwidth}\raggedright\strut
Number of Games Discarded\strut
\end{minipage} & \begin{minipage}[b]{0.05\columnwidth}\raggedright\strut
Number of Games Significant\strut
\end{minipage}\tabularnewline
\midrule
\endhead
\begin{minipage}[t]{0.05\columnwidth}\raggedright\strut
Thompson\strut
\end{minipage} & \begin{minipage}[t]{0.05\columnwidth}\raggedright\strut
95\strut
\end{minipage} & \begin{minipage}[t]{0.05\columnwidth}\raggedright\strut
1\strut
\end{minipage} & \begin{minipage}[t]{0.05\columnwidth}\raggedright\strut
0\strut
\end{minipage} & \begin{minipage}[t]{0.05\columnwidth}\raggedright\strut
5\strut
\end{minipage}\tabularnewline
\begin{minipage}[t]{0.05\columnwidth}\raggedright\strut
-\strut
\end{minipage} & \begin{minipage}[t]{0.05\columnwidth}\raggedright\strut
2\strut
\end{minipage} & \begin{minipage}[t]{0.05\columnwidth}\raggedright\strut
3\strut
\end{minipage} & \begin{minipage}[t]{0.05\columnwidth}\raggedright\strut
4\strut
\end{minipage}\tabularnewline
\begin{minipage}[t]{0.05\columnwidth}\raggedright\strut
-\strut
\end{minipage} & \begin{minipage}[t]{0.05\columnwidth}\raggedright\strut
3\strut
\end{minipage} & \begin{minipage}[t]{0.05\columnwidth}\raggedright\strut
42\strut
\end{minipage} & \begin{minipage}[t]{0.05\columnwidth}\raggedright\strut
2\strut
\end{minipage}\tabularnewline
\bottomrule
\end{longtable}

    In order to quickly grade your table, we ask that you include the values
of the table in the cell below. \texttt{n\_discarded\_k*} is the number
of discarded observations due to undefined statistics, and
\texttt{n\_sig\_k*} is the number of significant observations where
\texttt{*} is the length of the conditioning set.

    \begin{Verbatim}[commandchars=\\\{\}]
{\color{incolor}In [{\color{incolor}54}]:} \PY{n}{n\PYZus{}discarded\PYZus{}k1} \PY{o}{=} \PY{l+m+mi}{0}
         \PY{n}{n\PYZus{}discarded\PYZus{}k2} \PY{o}{=} \PY{l+m+mi}{3}
         \PY{n}{n\PYZus{}discarded\PYZus{}k3} \PY{o}{=} \PY{l+m+mi}{42}
         \PY{n}{n\PYZus{}sig\PYZus{}k1} \PY{o}{=} \PY{l+m+mi}{5}
         \PY{n}{n\PYZus{}sig\PYZus{}k2} \PY{o}{=} \PY{l+m+mi}{4}
         \PY{n}{n\PYZus{}sig\PYZus{}k3} \PY{o}{=} \PY{l+m+mi}{2}
         
         \PY{c+c1}{\PYZsh{} YOUR CODE HERE}
         \PY{c+c1}{\PYZsh{}raise NotImplementedError()}
\end{Verbatim}


    \begin{Verbatim}[commandchars=\\\{\}]
{\color{incolor}In [{\color{incolor}55}]:} \PY{c+c1}{\PYZsh{} No moleste}
\end{Verbatim}


    \begin{Verbatim}[commandchars=\\\{\}]
{\color{incolor}In [{\color{incolor}56}]:} \PY{c+c1}{\PYZsh{} Yeah I\PYZsq{}m empty.  Wanna fight?}
\end{Verbatim}


    \subsection{Further Reading}\label{further-reading}

\textbf{ESPN reports on this type of analysis}\\
\emph{Haberstroh (2017)}. "He's heating up, he's on fire! Klay Thompson
and the truth about the hot hand".
http://www.espn.com/nba/story/\_/page/presents-19573519/heating-fire-klay-thompson-truth-hot-hand-nba

\textbf{PDFs included in this homework folder}\\
\emph{Daks, Desai, Goldberg (2018)}. "Do the GSW Have Hot Hands?"\\
\emph{Miller, Sanjurjo (2015)}. "Surprised by the Gambler's and Hot Hand
Fallacies? A Truth in the Law of Small Numbers"

We thank Alon Daks, Nishant Desai, Lisa Goldberg, and Alex Papanicolaou
for their contributions and suggestions in making this homework.

    \subsection{Submission}\label{submission}

You're almost done!

Before submitting this assignment, ensure that you have:

\begin{enumerate}
\def\labelenumi{\arabic{enumi}.}
\tightlist
\item
  Restarted the Kernel (in the menubar, select
  Kernel\(\rightarrow\)Restart \& Run All)
\item
  Validated the notebook by clicking the "Validate" button.
\end{enumerate}

Then,

\begin{enumerate}
\def\labelenumi{\arabic{enumi}.}
\tightlist
\item
  \textbf{Submit} the assignment via the Assignments tab in
  \textbf{Datahub}
\item
  \textbf{Upload and tag} the manually reviewed portions of the
  assignment on \textbf{Gradescope}
\end{enumerate}


    % Add a bibliography block to the postdoc
    
    
    
    \end{document}
